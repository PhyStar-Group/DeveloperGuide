\chapter{程序功能的说明}
\section{颗粒初始时刻分布设计}
同时,本程序可以计算颗粒团在湍流中的扩散作用。可以设计点,线,面,立方体,球,椭球等其他用户自定义图形。也可以由其他语言设计颗粒团,并生成相应的输入文档,最终导入到程序初始化过程中。
\begin{figure}[h]
	\centering
	\includegraphics[width=0.7\linewidth]{figure/part2_init_partterm}
	\caption{程序中可以生成的图案}
	\label{fig:part2initpartterm}
\end{figure}

具体实现步骤,\\
1)将init\_pattern.dat格式的文件放在执行文件的同一级目录下。文件格式如下
\begin{lstlisting}
variables ="x","y","z"
Zone I= 100000, f=point
1		0		1
0.5 	0.5 	1.2
......
\end{lstlisting}
2)将PartInfo中的颗粒格式np前添加标记“-”,即
\begin{lstlisting}
nt   np
6    -1000
......
\end{lstlisting}
注,PartInfo.in中的np的绝对值不能超过init\_pattern.dat中的颗粒总数。
\section{不可压缩均匀各向同性湍流单向并行程序Hitp\_v2.x}

\section{不可压缩槽道双向并行程序channel\_v2.x}

\section{可压缩槽道双向并行程序OPENCFD-PARTURB\_v1.x}
本程序的流体力学求解器使用的是中科院力学所李新亮老师的有限差分程序,而点颗粒求解器使用PARTURB颗粒部分。

\subsection{并行方向}
\begin{figure}[h]
	\centering
\usetikzlibrary{arrows}
\begin{tikzpicture}

\draw[dashed] (-5,3) edge (3,3);
\draw[dashed]  (-5,1) edge (3,1);
\draw[dashed]  (-5,-1) edge (3,-1);
\draw[dashed]   (-3,5) edge (-3,-1);
\draw[dashed]   (-1,5) edge (-1,-1);
\draw[dashed]   (1,5) edge (1,-1);
\draw[dashed]   (3,5) edge (3,-1);
\node (v15) at (4,5) {x方向};
\node (v16) at (-5,-2) {z方向};
\draw [-triangle 60] (-5,5) -- (v15);
\draw [-triangle 60] (-5,5)-- (v16);
\node at (-4,4) {(0,0)};
\node at (-2,4) {(0,1)};
\node at (0,4) {(0,2)};
\node at (2,4) {(0,3)};
\node at (-4,2) {(1,0)};
\node at (-2,2) {(1,1)};
\node at (0,2) {(1,2)};
\node at (2,2) {(1,3)};
\node at (-4,0) {(2,0)};
\node at (-2,0) {(2,1)};
\node at (0,0) {(2,2)};
\node at (2,0) {(2,3)};
\end{tikzpicture}
\end{figure}
\subsection{opencfd转parturb的接口程序}
在计算过程中,颗粒推进在流场推进之前,保证一次循环结束后,流场与颗粒场在同一时间。在本程序中,OPENCFD 与PARTURB唯一的公共代码空间在OCFD\_NS\_Solver.f90。PARTURB的相关代码在src\_particle 目录中。 其中为了方便移植,和避免编译器选项中添加-r8的设置,在程序中直接设置双精度变量。

