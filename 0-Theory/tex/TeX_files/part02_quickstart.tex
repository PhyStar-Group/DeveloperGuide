

\chapter{快速开始}
\section{不可压缩均匀各向同性湍流}
只针对双向并行版本hsit\_v1.x.x。
\subsection{$Re_\lambda=30$纯流场计算}
1)在工作目录中新建算例目录
\begin{lstlisting}
mkdir test-hit
cd test-hit
mkdir run
\end{lstlisting}
2) 将代码 hsit\_v1.x.x 拷贝至 test-hit 目录,并复制代码中的runscript脚本至当前目录或者认为设定的目录\\
\begin{lstlisting}
cp -r hsit_v1.x.x/runscript/* ./run/
\end{lstlisting}
3)在m\_para\_public.f90中设置流体全局变量,其中PROC\_r与PROC\_c分别对应行核数与列核数,对应$z$与$x$方向。minrc是PROC\_r与PROC\_c的最小公倍数(必须设置)。nx, ny, nz是三个方向网格数(保证网格数能被核数整除)
\begin{lstlisting}
parameter 	(PROC_r=2,PROC_c=2,PROC=(PROC_r)*(PROC_c)-1,minrc=2)	
!number of processors=PROC+1      
parameter    (nx=64,ny=64,nz=64)
\end{lstlisting}
4)对程序进行编译
\begin{lstlisting}
make
\end{lstlisting}
因为不同平台,编译器有所差异。 需要在makefile中对编译器参数进行适当的修改。
\begin{lstlisting}
F77   = mpiifort #此处可切换成mpif90或mpifort,根据平台编译器定
FLAGS = -shared-intel -mcmodel=medium -qopt-dynamic-align -r8 -O3 -132 #-i-dynamic -r8 -O3 -132 ##-shared-intel 这一选项不是必须的。 
\end{lstlisting}
5)复制执行文件至建立好的run文件中
\begin{lstlisting}
cp hsit.pro ../run/ 
\end{lstlisting}
6)进入run文件修改setting.in的程序控制参数

\begin{tabular}{|c|c|c|c|c|}
	\hline 
	参数名 &主要含义 &静态/动态  &参考值 &备注  \\ 
	\hline 
	step0 &起始步 &静态 &0&    \\ 
	\hline 
	stepn &在起始步后计算步数&静态&10000&    \\ 
	\hline 
	time &起始时间 &静态&0.0&    \\ 
	\hline 
	dt & 时间步长&动态&1e-3& \tabincell{c}{正数:定步长计算\\负数:变步长计算} \\ 
	\hline 
	idump &是否存瞬时场&动态&1&\tabincell{c}{ 1:存 \\0:不存,不影响续算文件\\4:存单精度文件}\\ 
	\hline 
	Ndump & 存瞬时场间隔 &动态&1000&    \\ 
	\hline 
	Nscreen &监控器输出间隔&动态 &10&    \\ 
	\hline 
	Nmovie &瞬时云图输出间隔&动态 &1000&    \\ 
	\hline 
	Nrefresh &更新参数间隔&动态 &100&    \\ 
	\hline 
	Nstat &更新统计间格&动态 &1000&    \\ 
	\hline 
	lx*pi &计算域x&静态&2&  单位是pi  \\ 
	\hline 
	ly*pi &计算域y&静态&2&  单位是pi   \\ 
	\hline 
	lz*pi &计算域z&静态&2&   单位是pi   \\ 
	\hline 
	iforce &是否加随即体积力&静态&1& 1:强迫湍流 0:衰减湍流   \\ 
	\hline
	 Re &雷诺数,粘性倒数&动态&30&  不是$Re_\lambda$  \\ 
	 \hline 
	shear&平均剪切&静态&0& 均匀剪切湍流中的设置\\ 
	\hline  	 
	iu &初始流场&静态&0& \tabincell{c}{0:随机流场 \\1:从mid文件续算,\\ >1:从iu步文件续算\\<-1:从-iu步的单精度文件续算} \\ 
		\hline
	fiber &是否计算颗粒&静态&0& 1:计算 0:不计算   \\ 
	\hline 
	couple &是否双向耦合&静态&0&\tabincell{c}{ 0:不计算 1:计算 \\-1:计算颗粒应力但不耦合}   \\ 
	\hline 
	ip &初始颗粒场&静态&0&\tabincell{c}{0:随机颗粒 \\1:从mid文件续算,\\>1:从ip步文件续算\\<-1:从-ip步的单精度文件续算}  \\ 
	\hline 
\end{tabular} 
7) 修改提交算例的脚本。包括核数与执行文件
\begin{lstlisting}
run #实验室小机器提交命令,使用前需要 chmod +x run
runvilje #挪威vilje 机群
runfram #挪威fram 机器
runfit.sh  #fit新测试系统
\end{lstlisting}

7)提交算例\\
因为不同系统的提交命令不同,具体参见各自机器的提交命令。如果在小机器,如修改run中的代码为
\begin{lstlisting}
mpirun -n 4 ./hitp.pro
\end{lstlisting}
然后执行
\begin{lstlisting}
./run
chmod +x run        #如果不能执行,尝试添加执行权限
setsid ./run >xx.out  #如果需要提交到后台计算
\end{lstlisting}
8)重新编译\\
如果未对m\_para\_public.f90 进行修改,可以直接编译。大部分系统忽略未修改的文件。
如果对m\_para\_public.f90 进行修改,需要删除除lib文件的所有执行文件,可以使用
\begin{lstlisting}
chmod +x clean
./clean 0    #清楚所有执行文件,不包括库文件
./clean 1   #清除所有执行文件
./clean 2   #清除所有执行文件, 包括runscript的测试编译文件
\end{lstlisting}
 \subsection{颗粒相的计算}
 一般情况下,待流场充分发展后即可加入点颗粒进行计算。\\
 1)修改PartInfo.in中的颗粒参数
 \begin{lstlisting}
nt   np    !!particle type number, the number per type
13   1e5  !
!!----the help for shapeID, pls see the end of this script----------!!
!!------------------------------------------------------------------!!  
nt	   ShapeID	  DensityR	 Radius	   ka	kb	 kc   gravity_direc	   gravity
1	    301	         1	     0.	       -1    1	 0.	         1	         0
2	    300	         1	     0.	        1	 1	 100	       1	         0
3	    300	         1	     0.	        1	 1	 50	         1	         0
4	    300	         1	     0.	        1	 1	 20	         1	         0
5	    300	         1	     0.	        1	 1	 10	         1	         0
6	    300	         1	     0.	        1	 1	 3	         1	         0
7	    300	         1	     0.	        1	 1	 1	         1	         0
8	    300	         1	     0.	        1	 1	 0.333       1	         0
9	    300	         1	     0.	        1	 1	  0.1	       1	         0
10	  300	         1	     0.	        1	 1	  0.05	     1	         0
11	  300	         1	     0.	        1	 1	  0.02	     1	         0
12	  300	         1	     0.	        1	 1	  0.01       1	         0
13	  301	         1	     0.	        1	-1	   0.	       1	         0  
        	          	          	          	          	          	          	          	          
Others	          	          	          	          	          	          	          	          
nt	   ShapeId	     vswim	   bgyro	          	    
!!**********************************************************************
!!ShapeID Help 
Sp=100:103 sphere:0-stokes drag 1-empirical drag 2-Edrag+saffman, 3-Edrag+Mclaughlin             
Sp=200:201 inertia tri elliposids; spheroid, must keep ka=kb=Radius, kc=c/Radius                 
Sp=300 ellipsoid tracer                                                                          
sp=301 general tracer  ka=Lamb1=-Lambda, kb=Lamb2=Lambda, kc=Lamb3=0. for passive vector        
sp=400 LCS, using Sp=4, let ka=Lamb1=1,kb=Lamb2=-1,kc=Lamb3=1;                                 
sp=500:501 inertialess elliposids swimmer , inertialess spheriod swimmer 
!!************************************************************************
 \end{lstlisting}
2)修改setting 中的参数 fiber为1即可。另外,因为刚加入颗粒相时,流场已经充分发展。此时,需要让iu=1,让流场从续算文件读取。而让ip=0,让颗粒场从随机分布开始。
\subsection{常规输出文件}
可以使用tail –f ../data/moniter 动态查看监控文件
\begin{lstlisting}
step:          70                                 #当前步
NS time:  0.113404035568237  					  #当前步流体求解时间   
particle time:  0.500000000000000E+000			  #当前步颗粒求解时间
moments  time:  0.000000000000000E+000			  #预留,暂时忽略
CPU time:  0.613404035568237     				  #当前步计算总耗时
physical time:  0.194641187929005     			  #当前步对应的物理时间
div=  2.331468351712829E-015					  #当前步最大散度
courant dt=  3.939217410225782E-003				  #当前步允许最大时间步长
particle transfer number 3.000000000000000E+000	  #当前步每个进程平均颗粒传输
\end{lstlisting}
可以使用tail –f ../data/energy 动态查看瞬时统计值
\begin{lstlisting}
time, energy,eps_v,Re_lambda,x_length,lambda,tau_e,eta,etakmax,dx/eta
\end{lstlisting}
常见的输出文件
\begin{lstlisting}
energyspec-xxxxxxx  瞬时能谱
Vel_yz xxxxxxx.xxx  瞬时流场yz 云图
Velocityfield2paraxxx.xxxxxxx 流场数据, 二进制
Velocityfield2paraxxx.mid     流场续算文件,二进制
Particle xxxxxxx.xxx.dat 瞬时颗粒输出
particlefield2paraxxx.xxxxxxx 颗粒数据, 二进制
particlefield2paraxxx.mid     颗粒续算文件,二进制
Partinfo.out    颗粒信息输出
Part_stat      颗粒tumbling转动统计
\end{lstlisting}

\subsection{核数调整与网格加密}
在很多情况下,我们需要对进程数和网格数进行修改。本程序并未在主程序中提供相关操作。但可以在前处理pre文件夹下使用pregrid.f90 对进程数和网格数修改。(暂不支持)
\begin{lstlisting}
暂不支持
\end{lstlisting}
基本步骤:\\
1)将旧文件的*.mid或者瞬时场文件拷贝进 pre 目录,mid文件需要将文件后缀改为 *.mid0 避免新文件将旧文件覆盖。\\
2)修改pregrid.f90 对应参数, 进行编译。\\
3)运行即可。\\

\section{不可压缩均匀剪切湍流}
与均匀剪切湍流相同,但需要令iforce=0并给出Shear的值。另外,剪切湍流的剪切边界条件依赖于时间,所以续算过程中,需要给出准确的起始时间。
\section{不可压缩槽道湍流}
与均匀各向同性湍流程序类似, 槽道程序也为空间双向并行。
\subsection{$Re_\tau=180$的纯流场计算}
1)在工作目录中新建算例目录
\begin{lstlisting}
mkdir test-channel
cd test-channel
mkdir run
\end{lstlisting}
2) 将代码 channel\_v2.x.x 拷贝至 test-channel 目录,并复制代码中的runscript脚本至当前目录或者认为设定的目录\\
\begin{lstlisting}
cp -r channel_v2.x.x/runscript/* ./run/
\end{lstlisting}

3)在m\_para\_public.f90中设置流体全局变量,其中PROC\_r代表并行行数,其中PROC\_c代表并行列数,分别对应$z$与$x$方向.nx, ny, nz是三个方向网格数(保证nx与nz均能被PROC\_R核PROC\_C整除,保证ny能被PROC\_R整除)。 minrc 代表PROC\_r与PROC\_c的最小公倍数(必须设置)。
\begin{lstlisting}
parameter 	(PROC_r=4,PROC_c=4,PROC=(PROC_r)*(PROC_c)-1,minrc=4)		!number of processors=PROC+1
!!保证nx与nz均能被PROC_R核PROC_C整除,保证ny能被PROC_R整除
parameter         (nx=96,ny=128,nz=96)
\end{lstlisting}
4)对程序进行编译
\begin{lstlisting}
make
\end{lstlisting}
因为不同平台,编译器有所差异。 需要在makefile中对编译器参数进行适当的修改。
\begin{lstlisting}
F77   = mpiifort #此处可切换成mpif90或mpifort,根据平台编译器定
FLAGS = -shared-intel -mcmodel=medium -qopt-dynamic-align -r8 -O3 -132 #-i-dynamic -r8 -O3 -132 ##-shared-intel 这一选项不是必须的。 
\end{lstlisting}
5)复制执行文件至建立好的run文件中
\begin{lstlisting}
cp chn.pro ../run/ 
\end{lstlisting}
6)修改setting.in的程序控制参数\\
\begin{tabular}{|c|c|c|c|c|}
	\hline 
	参数名 &主要含义 &静态/动态  &参考值 &备注  \\ 
	\hline 
	step0 &起始步 &静态 &0&    \\ 
	\hline 
	stepn &在起始步后计算步数&静态&10000&    \\ 
	\hline 
	time &起始时间 &静态&0.0&    \\ 
	\hline 
	dt & 时间步长&动态&1e-3& \tabincell{c}{正数:定步长计算\\负数:变步长计算\\重新跑流场,用变步长计算\\待计算稳定后,改定步长计算} \\ 
	\hline 
	idump &是否存瞬时场&动态&1&\tabincell{c}{ 1:存 \\0:不存,不影响续算文件\\4:存单精度文件}\\ 
	\hline 
	Ndump & 存瞬时场间隔 &动态&1000&    \\ 
	\hline 
	Nscreen &监控器输出间隔&动态 &10&    \\ 
	\hline 
	Nmovie &瞬时云图输出间隔&动态 &1000&    \\ 
	\hline 
	Nrefresh &更新参数间隔&动态 &100&    \\ 
	\hline 
	Nstat &统计间隔&动态 &1000&    \\ 
	\hline 
	lx*pi &计算域x&静态&2&  单位是pi  \\ 
	\hline 
	ly &计算域y&静态&2&     \\ 
	\hline 
	lz*pi &计算域z&静态&1&   单位是pi   \\ 
	\hline 
     stretch &法向网格拉伸系数&静态&0&\tabincell{c}{s<0:用反双曲正切 \\0<s<1:用反正弦函数,\\s>1:用反正切函数}  \\ 
     	\hline  
    constantflow &是否定流量计算&静态&0&\tabincell{c}{1:定流量计算 \\0:定压力计算}  \\ 
    \hline
    Re &壁面摩擦雷诺数&动态&30&  定压力计算时使用  \\ 
     \hline 
     Rem &基于平均流量的雷诺数&动态&2820&\tabincell{c}{定流量计算时用}  \\ 
     \hline 
	iu &初始流场&静态&0& \tabincell{c}{0:随机流场 \\1:从mid文件续算,\\ 其他:从iu步文件续算} \\ 
\hline 	
	fiber &是否计算颗粒&静态&0& 1:计算 0:不计算   \\ 
	\hline 
	couple &是否双向耦合&静态&0&\tabincell{c}{ 0:不计算 1:计算 \\-1:计算颗粒应力但不耦合}   \\ 
	\hline 
	ip &初始颗粒场&静态&0&\tabincell{c}{0:随机颗粒 \\1:从mid文件续算,\\其他:从ip步文件续算}  \\ 
	\hline 


\end{tabular} 
7) 修改提交算例的脚本。包括核数与执行文件
\begin{lstlisting}
run #实验室小机器提交命令,使用前需要 chmod +x run
runvilje #挪威vilje 机群
runfram #挪威fram 机器
runfit.sh  #fit新测试系统
\end{lstlisting}
8)提交算例\\
因为不同系统的提交命令不同,具体参见各自机器的提交命令。如果在小机器,如修改run中的代码为
\begin{lstlisting}
mpirun -n 16 ./channel.pro
\end{lstlisting}
然后执行
\begin{lstlisting}
./run
chmod +x run        #如果不能执行,尝试添加执行权限
setsid ./run >xx.out  #如果需要提交到后台计算
\end{lstlisting}
8)重新编译\\
如果未对m\_para\_public.f90 进行修改,可以直接编译。大部分系统忽略未修改的文件。
如果对m\_para\_public.f90 进行修改,需要删除除lib文件的所有执行文件,可以使用
\begin{lstlisting}
chmod +x clean
./clean 0    #清楚所有执行文件,不包括库文件
./clean 1   #清除所有执行文件
./clean 2   #清除所有执行文件,包括runscript中的测试文件
\end{lstlisting}
\subsection{颗粒相的计算}
一般情况下,待流场充分发展后即可加入点颗粒进行计算。\\
1)修改PartInfo.in中的颗粒参数
\begin{lstlisting}
nt   np    !!particle type number, the number per type
13   1e5  !
!!----the help for shapeID, pls see the end of this script----------!!
!!------------------------------------------------------------------!!  
nt	   ShapeID	  DensityR	 Radius	   ka	kb	 kc   gravity_direc	   gravity
1	    301	         1	     0.	       -1    1	 0.	         1	         0
2	    300	         1	     0.	        1	 1	 100	       1	         0
3	    300	         1	     0.	        1	 1	 50	         1	         0
4	    300	         1	     0.	        1	 1	 20	         1	         0
5	    300	         1	     0.	        1	 1	 10	         1	         0
6	    300	         1	     0.	        1	 1	 3	         1	         0
7	    300	         1	     0.	        1	 1	 1	         1	         0
8	    300	         1	     0.	        1	 1	 0.333       1	         0
9	    300	         1	     0.	        1	 1	  0.1	       1	         0
10	  300	         1	     0.	        1	 1	  0.05	     1	         0
11	  300	         1	     0.	        1	 1	  0.02	     1	         0
12	  300	         1	     0.	        1	 1	  0.01       1	         0
13	  301	         1	     0.	        1	-1	   0.	       1	         0  

Others	          	          	          	          	          	          	          	          
nt	   ShapeId	     vswim	   bgyro	          	    
!!**********************************************************************
!!ShapeID Help 
Sp=100:103 sphere:0-stokes drag 1-empirical drag 2-Edrag+saffman, 3-Edrag+Mclaughlin             
Sp=200:201 inertia tri elliposids; spheroid, must keep ka=kb=Radius, kc=c/Radius                 
Sp=300 ellipsoid tracer                                                                          
sp=301 general tracer  ka=Lamb1=-Lambda, kb=Lamb2=Lambda, kc=Lamb3=0. for passive vector        
sp=400 LCS, using Sp=4, let ka=Lamb1=1,kb=Lamb2=-1,kc=Lamb3=1;                                 
sp=500:501 inertialess elliposids swimmer , inertialess spheriod swimmer 
!!************************************************************************
\end{lstlisting}
2)修改setting 中的参数 fiber为1即可。另外,因为刚加入颗粒相时,流场已经充分发展。此时,需要让iu=1,让流场从续算文件读取。而让ip=0,让颗粒场从随机分布开始。
\subsection{常规输出文件}
可以使用tail –f ../data/moniter 动态查看监控文件
\begin{lstlisting}
step:          70                                 #当前步
NS time:  0.113404035568237  					  #当前步流体求解时间   
particle time:  0.500000000000000E+000			  #当前步颗粒求解时间
moments  time:  0.000000000000000E+000			  #预留,暂时忽略
CPU time:  0.613404035568237     				  #当前步计算总耗时
physical time:  0.194641187929005     			  #当前步对应的物理时间
div=  2.331468351712829E-015					  #当前步最大散度
courant dt=  3.939217410225782E-003				  #当前步允许最大时间步长
particle transfer number 3.000000000000000E+000	  #当前步每个进程平均颗粒传输
\end{lstlisting}
可以使用tail –f ../data/bulk 动态查看瞬时统计值
\begin{lstlisting}
time, u_\tau, bulk, Re_\tau, visc
\end{lstlisting}
常见的输出文件
\begin{lstlisting}
energyspec-xxxxxxx  瞬时能谱
Vel_yz xxxxxxx.xxx  瞬时流场yz 云图
Velocityfield2paraxxx.xxxxxxx 流场数据, 二进制
Velocityfield2paraxxx.mid     流场续算文件,二进制
Particle xxxxxxx.xxx.dat 瞬时颗粒输出
particlefield2paraxxx.xxxxxxx 颗粒数据, 二进制
particlefield2paraxxx.mid     颗粒续算文件,二进制
Partinfo.out    颗粒信息输出
Part_stat      颗粒tumbling转动统计
\end{lstlisting}
\subsection{核数调整}
在很多情况下,我们需要对进程数进行修改。本程序并未在主程序中提供相关操作。但可以在前处理pre文件夹下使用pregrid\_d.f90 对进程数修改。
\begin{lstlisting}
integer, parameter:: PROC_r_old=4, PROC_c_old=4,PROC_old=PROC_r_old*PROC_c_old-1 !旧核数
integer, parameter:: PROC_r_new=16,PROC_c_new=2,PROC_new=PROC_r_new*PROC_c_new-1 !新核数
integer, parameter:: PROC_slice=10-1 !法向临时划分份数
integer, parameter:: nx_old =96, ny_old = 128, nz_old =96 !旧网格数
integer, parameter:: nx_new  =96, ny_new  =128, nz_new  =96 !新网格数
integer, parameter:: istep_old=444, istep_new=666 ! 对于瞬时场文件
integer, parameter:: iu=1 !1- mid 2-瞬时场
real(kind=r8),  parameter:: pi=4.*atan(1.),Retau=180, Rem=2820  
real(kind=r8),  parameter:: lx=2.*pi, ly=2., lz=1.*pi 
real(kind=r8),  parameter:: stretch_old =-1.5, stretch_new=-1.5   
integer, parameter:: iselect=1!! 1-网格调整  
\end{lstlisting}
基本步骤:\\
1)将旧文件的*.mid或者瞬时场文件拷贝进 pre 目录,mid文件需要将文件后缀改为 *.mid0 避免新文件将旧文件覆盖。\\
2)修改pregrid\_d.f90 对应参数, 进行编译。\\
3)运行即可。\\
\section{可压缩槽道OPENCFD-PARTURB}
\subsection{用opencfd计算$Re_\tau=180$的槽道}
1)在工作目录中新建算例目录
\begin{lstlisting}
mkdir test-compress
cd test-compress
mkdir run
cp -r test-compress/runscript/* run
\end{lstlisting}
2) 将代码 opencfd\_parturb\_x.x 拷贝至 test-compress 目,并进入code目录\\
3)对程序进行编译
\begin{lstlisting}
make
\end{lstlisting}
因为不同平台,编译器有所差异。 需要在makefile中对编译器参数进行适当的修改。
\begin{lstlisting}
F77   = mpiifort #此处可切换成mpif90或mpifort,根据平台编译器定 
\end{lstlisting}
4)复制执行文件到run目录
\begin{lstlisting}
cp *.pro ../run
\end{lstlisting}
5)run目录下opencfd.in的程序控制参数,在此只对部分变量进行简要说明,详细说明见opencfd使用手册。\\
\begin{tabular}{|c|c|c|c|}
	\hline 
	参数名 &主要含义 &  参考值 &备注  \\ 
	\hline 
	 nx &x方向网格数 &192&    \\ 
	\hline 
	 ny &y方向网格数 &192&    \\ 
	\hline
	 nz &z方向网格数 &192&    \\ 
	\hline
	npx0 &x方向并行核数 &4&    \\ 
	\hline 
	npy0 &y方向并行核数 &1& 该方向并行核数必须为1 \\ 
	\hline
	npz0 &z方向并行核数 &4&    \\ 
	\hline
	Iflag\_grid &网格设置 &0,1,0&    \\ 
	\hline
	Iperiodic &是否周期边界条件&1,0,1&    \\ 
	\hline
	SLx &x方向计算域 &2pi&    \\ 
	\hline 
	SLy &y方向计算域 &2&    \\ 
	\hline
	SLz &z方向计算域 &2pi&    \\ 
	\hline
	Re &基于平均速度的雷诺数 &2800&    \\ 
	\hline
	Ama &马赫数 &0.3&    \\ 
	\hline
	dt &时间步长 &4e-4&    \\ 
	\hline
	end\_time &结束时间 &1000&    \\ 
	\hline
	Kstep\_save &瞬时场存储间隔 &1000& \\ 
	\hline
\end{tabular}
 
 同时,由于不计算颗粒相,需要让Partinfo.in中的fiber=0.
6)初始流场生成\\
修改grid.in, 让网格与opencfd.in匹配。
\begin{lstlisting}
chmod +x initmake
./initmake
\end{lstlisting}
7)提交算例\\
因为不同系统的提交命令不同,具体参见各自机器的提交命令。如果在小机器,如修改run中的代码为
\begin{lstlisting}
mpirun -n 16 ./xxxx #xxxx是执行文件名
\end{lstlisting}
然后执行
\begin{lstlisting}
./run
chmod +x run        #如果不能执行,尝试添加执行权限
setsid ./run >xx.out  #如果需要提交到后台计算
\end{lstlisting}
8)重新编译\\
\begin{lstlisting}
chmod +x clean
./clean 0    #清除src_particle下所有执行文件
./clean 1   #清除所有执行文件
\end{lstlisting}
\subsection{颗粒相的计算}
一般情况下,待流场充分发展后即可加入点颗粒进行计算。\\
1)修改PartInfo.in中的控制参数和颗粒参数
\begin{lstlisting}
fiber ip Ndump Nscreen Nmovie
1     0   1000   10     1000
nt   np    !!particle type number, the number per type
13   1e5  !
!!----the help for shapeID, pls see the end of this script----------!!
!!------------------------------------------------------------------!!  
nt	   ShapeID	  DensityR	 Radius	   ka	kb	 kc   gravity_direc	   gravity
1	    301	         1	     0.	       -1    1	 0.	         1	         0
2	    300	         1	     0.	        1	 1	 100	       1	         0
3	    300	         1	     0.	        1	 1	 50	         1	         0
4	    300	         1	     0.	        1	 1	 20	         1	         0
5	    300	         1	     0.	        1	 1	 10	         1	         0
6	    300	         1	     0.	        1	 1	 3	         1	         0
7	    300	         1	     0.	        1	 1	 1	         1	         0
8	    300	         1	     0.	        1	 1	 0.333       1	         0
9	    300	         1	     0.	        1	 1	  0.1	       1	         0
10	  300	         1	     0.	        1	 1	  0.05	     1	         0
11	  300	         1	     0.	        1	 1	  0.02	     1	         0
12	  300	         1	     0.	        1	 1	  0.01       1	         0
13	  301	         1	     0.	        1	-1	   0.	       1	         0  

Others	          	          	          	          	          	          	          	          
nt	   ShapeId	     vswim	   bgyro	          	    
!!**********************************************************************
!!ShapeID Help 
Sp=100:103 sphere:0-stokes drag 1-empirical drag 2-Edrag+saffman, 3-Edrag+Mclaughlin             
Sp=200:201 inertia tri elliposids; spheroid, must keep ka=kb=Radius, kc=c/Radius                 
Sp=300 ellipsoid tracer                                                                          
sp=301 general tracer  ka=Lamb1=-Lambda, kb=Lamb2=Lambda, kc=Lamb3=0. for passive vector        
sp=400 LCS, using Sp=4, let ka=Lamb1=1,kb=Lamb2=-1,kc=Lamb3=1;                                 
sp=500:501 inertialess elliposids swimmer , inertialess spheriod swimmer 
!!************************************************************************
\end{lstlisting}

\subsection{颗粒部分常规输出文件}
常见的输出文件
\begin{lstlisting}
Particle xxxxxxx.xxx.dat 瞬时颗粒输出
particlefield2paraxxx.xxxxxxx 颗粒数据, 二进制
particlefield2paraxxx.mid     颗粒续算文件,二进制
Partinfo.out    颗粒信息输出
Part_stat      颗粒统计
\end{lstlisting}

