\chapter{颗粒与流场双向耦合基本理论}

\section{力耦合}
\section{应力与力矩耦合}
根据\citet{batchelor_stress_1970}的推导\footnote{原文中的Jeffery力矩在左手系,本文推导一律右手系,因此力矩与原文相差一个负号},单个颗粒产生的颗粒应力的表达式为:
\begin{equation}\label{eq:ch3-stress}
\begin{split}
\sigma_{ij}^{(p)}&=\frac{4\pi\mu}{V_{box}} \frac{1}{2}(H_{ij}+H_{ji})+\frac{1}{2V_{box}} \epsilon_{ijk} M_k\\
&=\frac{1}{V_{box}}{ [ C_{ijkl}S_{kl}+C_{ijk} M_k -\frac{1}{2}\epsilon_{ijk} M_k]}
\end{split}
\end{equation}
其中$V_{box}$是欧拉网格体积,$M_k$是作用在颗粒上的合力矩(Jeffery 力矩)\footnote{与\citet{batchelor_stress_1970}文章中差负号}。将公式(\ref{eq:ch3-stress})转入到颗粒坐标系, 即得
\begin{equation}
\sigma_{i'j'}^{(p)}
=\frac{1}{V_{box}}{ [C_{i'j'k'l'}S_{k'l'}+C_{i'j'k'} M_{k'}-\frac{1}{2}\epsilon_{i'j'k'}M_{k'}]}
\end{equation}
各项的表达式为:
\begin{equation}
\begin{split}
 C_{i'j'k'l'}S_{k'l'}&=\frac{16\pi \mu abc}{3(J_1J_2+J_2J_3+J_3J_1)}\left[\begin{matrix}
 J_1 S_{1'1'}-\Delta & &\\
 &J_2 S_{2'2'}-\Delta &\\
 & &J_3 S_{3'3'}-\Delta\\
 \end{matrix} \right] \\
 &+\frac{16\pi \mu abc}{3}\left[\begin{matrix}
  &S_{1'2'}/I_3&S_{1'3'}/I_2\\
 S_{2'1'}/I_3& &S_{2'3'}/I_1\\
 S_{3'1'}/I_2&S_{3'2'}/I_1 &\\
 \end{matrix} \right] \\
 \end{split}
\end{equation}

\begin{equation}
\begin{split}
C_{i'j'k'}  M_{k'}&=\frac{1}{2}\left[\begin{matrix}
&\Theta_3 M_{3'}&\Theta_2M_{2'}\\
\Theta_3M_{3'}& &\Theta_1 M_{1'}\\
\Theta_2M_{2'}&\Theta_1M_{1'} &\\
\end{matrix} \right] \\
\end{split}
\end{equation}
\begin{equation}
\begin{split}
\tau_{ij} =-\frac{1}{2}\epsilon_{i'j'k'} M_{k'}&=-\frac{1}{2}\left[\begin{matrix}
&M_{3'}&-M_{2'}\\
-M_{3'}& &M_{1'}\\
M_{2'}&-M_{1'} &\\
\end{matrix} \right] \\
\end{split}
\end{equation}
其中形状参数为$\Theta_1=\frac{k_2^2-k_3^2}{k_2^2+k_3^2}$,$\Theta_2=\frac{k_3^2-k_1^2}{k_3^2+k_1^2}$,$\Theta_3=\frac{k_1^2-k_2^2}{k_1^2+k_2^2}$。$\Delta=\frac{1}{3}(J_1S_{1'1'}+J_2S_{2'2'}+J_3S_{3'3'})$同时 $I_i$和$J_i$分别为:
\begin{equation}
\begin{split}
I_{1}=\int_{0}^{\infty} \frac{k_1k_2k_3(k_2^2+k_3^2)d \lambda}{\Delta(\lambda)(k_2^2+\lambda)(k_3^2+\lambda)},\quad J_{1}=\int_{0}^{\infty} \frac{k_1k_2k_3\lambda d \lambda}{\Delta(\lambda)(k_2^2+\lambda)(k_3^2+\lambda)};\\
I_{2}=\int_{0}^{\infty} \frac{k_1k_2k_3(k_1^2+k_3^2)d \lambda}{\Delta(\lambda)(k_1^2+\lambda)(k_3^2+\lambda)},\quad J_{2}=\int_{0}^{\infty} \frac{k_1k_2k_3\lambda d \lambda}{\Delta(\lambda)(k_1^2+\lambda)(k_3^2+\lambda)};\\
I_{3}=\int_{0}^{\infty} \frac{k_1k_2k_3(k_1^2+k_2^2)d \lambda}{\Delta(\lambda)(k_1^2+\lambda)(k_2^2+\lambda)},\quad J_{3}=\int_{0}^{\infty} \frac{k_1k_2k_3\lambda d \lambda}{\Delta(\lambda)(k_1^2+\lambda)(k_2^2+\lambda)};\\
\end{split}
\end{equation}
其中$M_{i'}$ 与$\tilde{M}_{i’}$的关系
\begin{equation}\label{key}
\begin{split}
M_{1'}=\frac{16\pi \mu r^3}{3}(k_2^2+k_3^2)\tilde{M}_{1'}\\
M_{2'}=\frac{16\pi \mu r^3}{3}(k_3^2+k_1^2)\tilde{M}_{2'}\\
M_{3'}=\frac{16\pi \mu r^3}{3}(k_1^2+k_2^2)\tilde{M}_{3'}\\
\end{split}
\end{equation}
其中$D$为密度比,$r$为参考半径,$k_i$为长细比。具体标记见颗粒求解器基本理论。