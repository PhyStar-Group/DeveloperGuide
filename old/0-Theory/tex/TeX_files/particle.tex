\chapter{颗粒求解器基本理论}


\section{球形颗粒}
\begin{equation}
\begin{split}
m_p \frac{d v_i}{dt}&=F_i^d+F_i^{Saff}\pm (m_p-m_f)g_i\\
I_p \frac{d \omega_i}{dt}&=M_i^d
\end{split}
\end{equation}

\subsection{阻力模型}
\begin{itemize}
\item Stokes阻力: $\ve F^d=6\pi \mu R (\ve u-\ve v)$.
\item 经验阻力公式:$\ve F^d=6\pi \mu R (\ve u-\ve v)(1+0.15Re_p^{0.687})$.
\end{itemize}
\subsection{Saffman 升力}
\begin{equation}
\ve F^{Saff}=1.615J\mu2R|\Delta \ve u|\sqrt{\frac{4R^2|\ve \Omega|}{\nu}}\frac{\ve \Omega\times\Delta \ve u}{|\ve \Omega||\Delta \ve u|}
\end{equation}
 其中$\Delta \ve u=\ve v-\ve u$.
\begin{itemize}
\item 	对于Saffman model 1: $J=1$
\item	对于Saffman model 2: $J=0.3(1+\tanh[2.5(\log_{10}\epsilon+0.191)])(\frac{2}{3}+\tanh(6\epsilon-1.92))$, 其中$\epsilon=\sqrt{|\ve \Omega|\nu}{\Delta \ve u}$.
\end{itemize}
\section{椭球颗粒}
\subsection{非球形颗粒绕流Stokes解的一般形式}
对于一般非球形的颗粒的绕流问题,其颗粒的受力可以描述为阻力,力矩与应力子。 它们的表达式分别为:
\begin{equation}{\label{eq:part01-stokes}}
\begin{split}
F_i &= K_{ij}^t(U_j-v_j)+K_{ij}^c(\Omega_j-\omega_j)+\Phi_{ijk}S_{jk}\\ 
M_i &= K_{ij}^c(U_j-v_j)+K_{ij}^r(\Omega_j-\omega_j)+\Theta_{ijk}S_{jk}\\
\Sigma_{ij}&=\Phi_{kij}(U_k-v_k)+\Theta_{kij}(\Omega_k-\omega_k)+Q_{ijkl}S_{kl}\\
\end{split}
\end{equation}
其中$K_{ij}^t, K_{ij}^C,K_{ij}^r,\Theta_{ijk},\Phi_{ijk},Q_{ijkl}$均与颗粒形状有关。同时,由于应力子的对称性,这些系数满足一些对称特点,即
\begin{equation}
\begin{split}
K_{ij}^t&=K_{ji}^t,\quad K_{ij}^c=K_{ji}^c,\quad K_{ij}^r=K_{ji}^r,\\ 
\Phi_{ijk}&=\Phi_{ikj},\quad \Theta_{ijk}=\Theta_{ikj},\\
Q_{ijkl}&=Q_{ijlk},\quad Q_{ijkl}=Q_{ijlk},\quad Q_{ijkl}=Q_{klij},\quad Q_{ijkl}=Q_{jilk}.\\
\end{split}
\end{equation}
通过公式(\ref{eq:part01-stokes})可知,对于一般形状的颗粒,颗粒在流场中受到的合力,合力矩是相互耦合,且应力子也与颗粒的平动相关。但对于中心对称颗粒,$K_{ij}^c=0$, $\phi_{ijk}=0$. 即:
   	\begin{equation}
\begin{split}
F_i &= K_{ij}^t(U_j-v_j)\\ 
M_i &= K_{ij}^r(\Omega_j-\omega_j)+\Theta_{ijk}S_{jk}\\
\Sigma_{ij}&=\Theta_{kij}(\Omega_k-\omega_k)+Q_{ijkl}S_{kl}.\\
\end{split}
\end{equation}
理论上,如果我们能获得任意形状颗粒的这些参数的具体表达式或者数值,我们便可以将其用于点颗粒计算。
\subsection{姿态描述}
对于非球形颗粒,颗粒的取向需要考虑。在此,我们引入颗粒的姿态描述。在介绍姿态描述之前,我们需要定义惯性坐标系,随体坐标系和颗粒坐标系,并阐述两种坐标系之间的联系。在本程序中,默认使用椭球颗粒模型。
\begin{figure}[h]
		\centering
	\includegraphics[width=0.8\linewidth]{Part1_frame.png}
	\caption{坐标系示意图}
\end{figure}
基矢量转换关系:
$\ve g_{i'}=R_{i'i} \ve g_{i}$, $\ve g_i=R_{ii'} \ve g_{i'}$ 。 其中 $\ve g_i$ 惯性坐标系的基矢量,
$\ve g_{i'}$  颗粒坐标系的基矢量。\\
 将惯性空间的矢量在颗粒坐标系中描述: $\ve n= n_i \ve g_i =n_i R_{ii'} \ve g_{i'}$ \\ 
 将颗粒空间的矢量在惯性坐标系中描述:$\ve n'= n_{i'} \ve g_{i'} =n_{i'} R_{i'i} \ve g_{i}$\\
 将惯性空间的张量在颗粒坐标系中描述: $\ma K= K_{ij} \ve g_i \ve g_j =K_{ij} R_{ii'}R_{jj'} \ve g_{i'}\ve g_{j'}$ \\ 
 将颗粒空间的张量在惯性坐标系中描述:$\ma K'= K_{i'j'} \ve g_{i'} \ve g_{j'} =K_{i'j'} R_{i'i}R_{j'j} \ve g_{i}\ve g_{j}$\\
\subsubsection{欧拉角与卡丹角}
\subsubsection{欧拉四元数}
在数值计算中,欧拉角和卡丹角容易产生奇异性。 但我们可以通过欧拉四元数来表征颗粒的取向。对于欧拉四元数的数学原理详见\obs{REF}
\begin{equation}
e_{0}^{2}+e_{1}^{2}+e_{2}^{2}+e_{3}^{2}=1
\end{equation}
\begin{equation}
\ma R_{i'i}=\left[ \begin{array}{ccc}{e_{0}^{2}+e_{1}^{2}-e_{2}^{2}-e_{3}^{2}} & {2\left(e_{1} e_{2}+e_{0} e_{3}\right)} & {2\left(e_{1} e_{3}-e_{0} e_{2}\right)} \\ {2\left(e_{1} e_{2}-e_{0} e_{3}\right)} & {e_{0}^{2}-e_{1}^{2}+e_{2}^{2}-e_{3}^{2}} & {2\left(e_{2} e_{3}+e_{0} e_{1}\right)} \\ {2\left(e_{1} e_{3}+e_{0} e_{2}\right)} & {2\left(e_{2} e_{3}-e_{0} e_{1}\right)} & {e_{0}^{2}-e_{1}^{2}-e_{2}^{2}+e_{3}^{2}}\end{array}\right]
\end{equation}

$\ve e =[e_0~e_1~e_2~e_3]^T$
\begin{equation}
\mathbf{G}=\left[ \begin{array}{cccc}{-e_{1}} & {e_{0}} & {e_{3}} & {-e_{2}} \\ {-e_{2}} & {-e_{3}} & {e_{0}} & {e_{1}} \\ {-e_{3}} & {e_{2}} & {-e_{1}} & {e_{0}}\end{array}\right]
\end{equation}
\begin{equation}
\dot{\mathbf{e}}=\frac{1}{2} \mathbf{G}^{T} \boldsymbol{\omega}'
\end{equation}

\subsection{平动模型}
对于一般性的椭球颗粒,其平动方程为
\begin{equation}\label{eq:ch2-etranslation}
m_{p} \frac{d v_{i}}{d t}= \mu  K_{i j} \Delta u_{j} \pm (m_p-m_f)\ve g
\end{equation}
其中$K_{i j}=R_{i i'}K_{i' j'} R_{j' j}$,取决于颗粒的取向. 其任意椭球体的Stokes
阻力模型\citep{brenner_stokes_1963}:
\begin{equation}
\mathbf{K'}=16 \pi\left[\frac{\ve {g}_{1} \ve {g}_{1}}{\chi+a_{1}^{2} \alpha_{1}}+\frac{\ve{g}_{2} \ve{g}_{2}}{\chi+a_{2}^{2} \alpha_{2}}+\frac{\ve{g}_{3} \ve{g}_{3}}{\chi+a_{3}^{2} \alpha_{3}}\right]
\end{equation}
为了方便表述,$\ve g_i$代表固结在颗粒坐标系上的基矢量。其中$a_i$($i=1,2,3$) 分别代表椭球三个主轴的半轴长,而$\alpha_{i}$和$\chi$均代表椭球体的形状因子。其各自的表达式为
\begin{equation}
\chi=\int_{0}^{\infty} \frac{d \lambda}{\Delta(\lambda)},
\end{equation}
\begin{equation}
\alpha_{i}=\int_{0}^{\infty} \frac{d \lambda}{\left(a_{j}^{2}+\lambda\right) \Delta(\lambda)} \quad(j=1,2,3).
\end{equation}
其中
\begin{equation}
\Delta(\lambda)=\sqrt{\left(a_{1}^{2}+\lambda\right)\left(a_{2}^{2}+\lambda\right)\left(a_{3}^{2}+\lambda\right)},
\end{equation}
将平动模型化简后,
\begin{equation}
 \frac{d v_{i}}{d t}= \frac{12 \nu}{D r^2 k_1k_2k_3}  \tilde{K}_{i' j'}R_{i' i} R_{j'j} \Delta u_{j} \pm (1-\frac{1}{D}) g_i
\end{equation}
其中$D$为颗粒与流体的密度比$\rho_p/\rho_f$,$k_i$为颗粒的长细比,即 $k_i=a_i/r$。 $r$为参考半径。
而无量纲阻力矩阵$\mathbf{\tilde{K}'}$为
\begin{equation}
\mathbf{\tilde{K}'}=\left[\frac{\ve{g}_{1} \ve{g}_{1}}{\tilde{\chi}+k_{1}^{2} \tilde{\alpha}_{1}}+\frac{\ve{g}_{2} \ve{g}_{2}}{\tilde{\chi}+k_{2}^{2} \tilde{\alpha}_{2}}+\frac{\ve{g}_{3}\ve{g}_{3}}{\tilde{\chi}+k_{3}^{2} \tilde{\alpha}_{3}}\right]
\end{equation}

而进行归一化的形状因子分别表示为:
\begin{equation}
\begin{split}
\tilde {\alpha}&=\int_{0}^{\infty} \frac{d \lambda}{\Delta(\lambda)(k_1^2+\lambda)},\quad \tilde {\beta}=\int_{0}^{\infty} \frac{ d \lambda}{\Delta(\lambda)(k_2^2+\lambda)};\\
\tilde {\gamma}&=\int_{0}^{\infty} \frac{d \lambda}{\Delta(\lambda)(k_3^2+\lambda)},\quad \tilde {\chi}=\int_{0}^{\infty} \frac{ d \lambda}{\Delta(\lambda)};\\
\end{split}
\end{equation}

对于回转体的椭球颗粒,令$r=a_1$, $a_1=a_2$.则有$k_1=k_2=1, k_3=\lambda$ (这里的$\lambda$ 指的长轴与短轴之比).

此时椭球体的形状参数有解析解:

\begin{tabular}{|c|c|c|c|}
	\hline 
	& 碟状颗粒 & 球形颗粒 & 杆状颗粒\\
	\hline
	& $\lambda<1$ & $\lambda=1$ & $\lambda>1$ \\ 
	\hline 
	$\tilde{\alpha}_1=\tilde{\alpha}_2$&$-\frac{B-\pi}{2(1-\lambda^2)^{3/2}}-\frac{\lambda}{(1-\lambda^2)}$& 2/3 &  $\frac{-A}{2(\lambda^2-1)^{3/2}}+\frac{\lambda}{(\lambda^2-1)}$\\ 
	\hline 
	$\tilde{\alpha}_3$& $\frac{B-\pi}{(1-\lambda^2)^{3/2}}+\frac{2}{(1-\lambda^2)\lambda}$ &2/3&  $\frac{A}{(\lambda^2-1)^{3/2}}-\frac{2}{\lambda(\lambda^2-1)}$ \\ 
	\hline 
	$\tilde{\chi}$&$-\frac{B-\pi}{(1-\lambda^2)}$&2& $\frac{A}{\sqrt{\lambda^2-1}}$  \\ 		
	\hline 
\end{tabular} 

其中 $A=2\ln {(\lambda+\sqrt{\lambda^2-1})}$ 和 $B=2\arctan{\frac{\lambda}{\sqrt{1-\lambda^2}}}$

\subsection{转动模型}

椭球颗粒转动动力学模型\citep{jeffery_motion_1922,lundell_heavy_2010,lundell_effect_2011}:

\begin{equation}
\ma{I'} \cdot \dot{\ve\omega'}+ \ve \omega' \times (\ma I' \cdot \ve \omega')=\ve M'
\end{equation}
其中$\prime$代表颗粒坐标系。颗粒受到的合外力矩为$\ve M'$
\begin{equation}
	\begin{split}
M_{1'}=\frac{16\pi\mu}{3(a_2^2 \alpha_2+a_3^2 \alpha_3)}\left[ (a_2^2-a_3^2)S_{2'3'}+(a_2^2+a_3^2)(\Omega_{3'2'}-\omega_{1'})\right] \\
M_{2'}=\frac{16\pi\mu}{3(a_3^2 \alpha_3+a_1^2 \alpha_1)}\left[ (a_3^2-a_1^2)S_{1'3'}+(a_3^2+a_1^2)(\Omega_{1'3'}-\omega_{2'})\right] \\
M_{3'}=\frac{16\pi\mu}{3(a_1^2 \alpha_1+a_2^2 \alpha_2)}\left[ (a_1^2-a_2^2)S_{1'2'}+(a_1^2+a_2^2)(\Omega_{2'1'}-\omega_{3'})\right] \\
	\end{split}
\end{equation}
写成基于角速度的微分方程形式
\begin{equation}
	\begin{split}
\frac{d \omega_{1'}}{d t}=\frac{20 \nu}{D r^2 k_1k_2k_3}\tilde{M}_{1'}-\Theta_1 \omega_{2'}\omega_{3'}\\
\frac{d \omega_{2'}}{d t}=\frac{20 \nu}{D r^2 k_1k_2k_3}\tilde{M}_{2'}-\Theta_2 \omega_{1'}\omega_{3'}\\
\frac{d \omega_{3'}}{d t}=\frac{20 \nu}{D r^2 k_1k_2k_3}\tilde{M}_{3'}-\Theta_3 \omega_{1'} \omega_{2'}\\
	\end{split}
\end{equation}
为此,定义新的形状参数为$\Theta_1=\frac{k_2^2-k_3^2}{k_2^2+k_3^2}$,$\Theta_2=\frac{k_3^2-k_1^2}{k_3^2+k_1^2}$,$\Theta_3=\frac{k_1^2-k_2^2}{k_1^2+k_2^2}$,则对应的无量纲力矩为
\begin{equation}
\begin{split}
\tilde{M}_{1'}&=\frac{1}{(k_2^2 \tilde{\alpha}_2+k_3^2 \tilde{\alpha}_3)}\left[ \Theta_1 S_{2'3'}+(\Omega_{3'2'}-\omega_{1'})\right] \\
\tilde{M}_{2'}&=\frac{1}{(k_3^2 \tilde{\alpha}_3+k_1^2 \tilde{\alpha}_1)}\left[ \Theta_2 S_{1'3'}+(\Omega_{1'3'}-\omega_{2'})\right] \\
\tilde{M}_{3'}&=\frac{1}{(k_1^2 \tilde{\alpha}_1+k_2^2 \tilde{\alpha}_2)}\left[ \Theta_3 S_{1'2'}+(\Omega_{2'1'}-\omega_{3'})\right] \\
\end{split}
\end{equation}
\subsection{颗粒响应时间}
考虑颗粒取向随机,定义颗粒的平动响应时间为\citep{brenner_stokes_1963,shin_rotational_2005}
\begin{equation}
\tau_p=\frac{Dr^2k_1k_2k_3}{12\nu}\frac{1}{\bar{\tilde{K}}}
\end{equation}
其中$\bar{\tilde{K}}$
\begin{equation}
\bar{\tilde{K}}=3(\tilde{K}_{1'1'}^{-1}+\tilde{K}_{2'2'}^{-1}+\tilde{K}_{3'3'}^{-1})^{-1}
\end{equation}
对于回转椭球体,其颗粒响应时间有解析解\citep{shapiro_deposition_1993}(注:$\lambda=a/b$, 其中$a$为长半轴,$b$为短半轴。表达式取决于颗粒的长细比定义方式。)
\begin{equation}
\tau_{p,prolate}=\frac{Da^2\lambda}{9\nu}\frac{2\ln{(\lambda+\sqrt{\lambda^2-1})}}{\sqrt{\lambda^2-1}}
\end{equation}
\begin{equation}
\tau_{p,oblate}=\frac{Da^2\lambda}{9\nu}\frac{\pi-2\arctan{\frac{\lambda}{\sqrt{1-\lambda^2}}}}{\sqrt{1-\lambda^2}}
\end{equation}
目前,暂时没有颗粒转动的响应时间的估计公式。
\section{弱惯性椭球颗粒模型}
当颗粒尺寸和密度足够小(大于或等于流体密度), 一般的颗粒运动模型的加速项计算十分困难。 在此,我们认为颗粒的合外力为零。因此,可以得出弱惯性椭球颗粒的运动控制方程。
\subsection{平动模型}
当$\frac{d v}{dt}=0$, 方程~(\ref{eq:ch2-etranslation})为
\begin{equation}
 \mu  K_{i j} \Delta u_{j} \pm (m_p-m_f)\ve g = 0
\end{equation}
因此,颗粒的速度将由两部分组成
\begin{equation}
\begin{split}
\ve v &=\ve u+\ve v_g\\
\ve v_g&=\pm\frac{D r^2 k_1k_2k_3}{12 \nu}(1-\frac{1}{D}) \ma {K}^{-1} \ve g
\end{split}
\end{equation}
其中$\ve v_g$为 Stokes 沉降速度, $\ve u$ 为颗粒处的流体速度。当不考虑沉降时,颗粒完全跟随流体。
\subsection{转动模型}
同理, 让$\frac{d \ve \omega'}{dt}=0$, 颗粒的转动速度为
\begin{equation}
\begin{split}
\omega_{1'}&=\Theta_1 S_{2'3'}+\Omega_{3'2'} \\
\omega_{2'}&=\Theta_2 S_{1'3'}+\Omega_{1'3'} \\
\omega_{3'}&=\Theta_3 S_{1'2'}+\Omega_{2'1'} \\
\end{split}
\end{equation}
\section{游动颗粒模型}

\section{pFTLE颗粒模型}
此类颗粒的类型主要用来计算随流体轨迹线的李雅普诺夫指数和拉格朗日拉伸与压缩方向。本颗粒不直接求解变形梯度张量,而是直接求解左柯西格林张量的特征向量与特征值进行近似计算。具体的概念与实现过程,可见Cui et al.(Draft).在此,仅简要给出公式。
	\begin{equation}\label{eq:angular velocity}
\begin{split}
\tilde{\omega}_1&= \tilde{\Omega}_{1} +\epsilon_1\tilde{S}_{23},\\
\tilde{\omega}_2&= \tilde{\Omega}_{2}+ \epsilon_2\tilde{S}_{31},\\
\tilde{\omega}_3&= \tilde{\Omega}_{3} +\epsilon_3\tilde{S}_{12}.
\end{split}
\end{equation}
其中
	\begin{equation}
\epsilon_1=\frac{e^{2\rho_2}+e^{2\rho_3}}{e^{2\rho_2}-e^{2\rho_3}},\epsilon_2=\frac{e^{2\rho_3}+e^{2\rho_1}}{e^{2\rho_3}-e^{2\rho_1}},\epsilon_3=\frac{e^{2\rho_1}+e^{2\rho_2}}{e^{2\rho_1}-e^{2\rho_2}}.
\end{equation}
而各个向量方向上的特征值的时间推进为:
\begin{equation}{\label{eq:rho_t}}
\frac{d \rho_i}{d t} = \tilde{S}_{ii}
\end{equation}
\section{颗粒处流场信息插值方法}
\subsection{三维二阶拉格朗日插值}
本程序的颗粒处的插值函数使用的是二阶拉格朗日插值。(注:在算有壁面问题时,壁面附近的网格不能太密,比如用$\cos\frac{2\pi j}{N}$,此网格在算高雷诺数问题时,插值容易出现龙格现象。)
\subsubsection{均匀网格方向}
\subsubsection{非均匀网格方向}
\section{时间推进方法}
\subsection{二阶显式Adams-Bashforth方法}
对于常微分方程组$\dot{f}=g$。 
\begin{equation}
\frac{f^{n+1}-f^{n}}{\Delta t}=\frac{3}{2}g^{n}-\frac{1}{2}g^{n-1}
\end{equation}

\section{颗粒碰撞}
