\documentclass[lang=cn,10pt,chinese,
cite=authoryear, plain]{elegantbook}

\title{PhyStar理论手册}
%\subtitle{Elegant\LaTeX{} 经典之作}

\author{PhyStar项目组}
%\institute{PhyStar项目组}
\date{\today}
%\version{3.10}
\bioinfo{邮箱}{czw17@tsinghua.org.cn}


%\extrainfo{}

%\logo{logo-blue.png}
\cover{cover.jpg}

% 本文档命令
\usepackage{array}
\newcommand{\ccr}[1]{\makecell{{\color{#1}\rule{1cm}{1cm}}}}
\newcommand\ve[1]{\bm {#1}}
\newcommand\ma[1]{\mathbf{#1}}
\newcommand\obs[1]{\textcolor{red}{#1}}
\newcommand{\tabincell}[2]{\begin{tabular}{@{}#1@{}}#2\end{tabular}}
\newcommand\Partial[2]{\frac{\partial #1}{\partial #2}}
% 修改目录深度
\setcounter{tocdepth}{2}

\begin{document}

\maketitle
\frontmatter


\chapter*{前言}
\markboth{Introduction}{前言}
PhyStar是一套面向多物理场科学计算的C++库。 名称
PhyStar由Physics(物理)和Star(星)结合而来,寓意物理之星,中文名为“星曜”。PhyStar将主要致力于多物理场问题(流体、固体、电磁、热化学、多组分和多相等)的求解,提供丰富且简易的调用接口。PhyStar将邀请相关领域优秀的年轻科学家或相关领域从业者进行合作开发与应用,致力于成为一款具有广泛影响力的计算库。 由于PhyStar初创团队长期从事流体力学以及颗粒湍流两相流的研究,因此,程序的大部分内容以及理论手册中大篇幅内容将与此有关,且大部分内容延续初创团队开发的Fortran程序PARTURB的说明文档。未来,我们希望会有更加丰富的内容进行添加,一方面作为程序的理论手册,另一方面成为前沿科学工作者的理论手册集合。

作者非常感谢Elegant\LaTeX{} 项目组提供的Latex模板。



\vskip 1.5cm

\begin{flushright}
作者\\
2023 年6月 写于清华园
\end{flushright}

\clearpage
\section*{本手册的贡献者}
\begin{tabular}{cccc}
	\hline 
	姓名 &单位 &时间&主要贡献  \\ 
	\hline 
	崔智文 &清华大学 &2017-至今 &手册主体内容 \\  
	\hline 

\end{tabular} 


\vskip 1.5cm


\tableofcontents
%\listofchanges

\mainmatter
\part{流体部分}
\chapter{流体模拟理论}
\section{基本方程}
\subsection{不可压缩湍流}
\begin{equation}\label{eq:ch1-ns}
\begin{split}
\frac{\partial \ve {u}}{\partial t}+\ve{u} \cdot \nabla \ve{u}&=-\nabla p / \rho+\nu \nabla^{2} \ve{u}+\ve{f}\\
\nabla \cdot \ve{u}&=0
\end{split}
\end{equation}
其中$\ve u$是湍流脉动速度, $p$ 是压强脉动, $\nu$ 是运动粘度, $\ve f$是体积力.

介绍其它可能形式
\subsection{可压缩湍流}

\subsection{大涡模拟}

\section{时间离散}
\subsection{模拟方法}
参考Challabotla 博士论文的方法,即
\begin{equation}
	\frac{u_i^{n+1}-u_i^n}{\Delta t}=\frac{3}{2}T(u_i^n)-\frac{1}{2}T(u_i^{n-1})-\frac{1}{\rho}\frac{\partial p^{n+1}}{ \partial x_i}
\end{equation}
其中
\begin{equation}
	T(u_i^n)=-u_j^n\frac{\partial u_i^n}{\partial x_j}+\nu\frac{\partial^2 u_i^n}{\partial x_j \partial x_j}
\end{equation}
求解步骤:\\
第一步:求解中间步 $u^*$并忽略压力$p^n$,
\begin{equation}
\frac{u_i^{*}-u_i^n}{\Delta t}=\frac{3}{2}T(u_i^n)-\frac{1}{2}T(u_i^{n-1})
\end{equation}
第二步: 求解压力步$p^{n+1}$,
\begin{equation}
\nabla^2p^{n+1}=\frac{\rho \nabla\cdot u^*}{\Delta t}
\end{equation}
第三步:求解速度$u_i^{n+1}$
\begin{equation}
u_i^{n+1}-u^*=-\Delta t\frac{\partial p^{n+1}}{\partial x_i}
\end{equation}

\subsection{空间离散}

\section{典型流体问题}
\subsection{均匀各向同性湍流}
\subsubsection{流场初始化方法}
采用\citet{rogallo_numerical_1981}初始化方法:
\begin{equation}
	\hat{\ve u}(\ve k)=\frac{1}{k\sqrt{k_1^2+k_2^2}}\left( \begin{array}{c}
	\alpha(k)kk_2+\beta(k)k_1k_3\\
	-\alpha(k)kk_1+\beta(k)k_2k_3\\
	-\beta(k)(k_1^2+k_2^2)
	\end{array}\right) 
\end{equation}
其中$k=\sqrt{k_1^2+k_2^2+k_3^2}$, $\alpha(k)=\sqrt{\frac{E(k)}{4\pi k^2}}\exp{i\theta_1}\cos\phi$, 而 $\beta(k)=\sqrt{\frac{E(k)}{4\pi k^2}}\exp{i\theta_2}\cos\phi$. $\theta_1$,$\theta_2$与$\phi$为[0,$2\pi$]之间的随机数。同时,我们也需要给定初始能谱$E(k)$, 在本程序中,默认能谱\citep{sullivan_deterministic_1994}
\begin{equation}
E(k)=\frac{q^2}{2A}\frac{k_p}{k_p ^{\sigma+1}}\exp{(-\frac{\sigma}{2}\frac{k}{k_p}^2)}
\end{equation}
\subsubsection{强迫各向同性湍流:加力方法}
\noindent \textbf{低波数加力}\\
只在大尺度部分添加能量,具体形式如下\citep{machiels_predictability_1997}
\begin{equation}
	\hat{\ve f}=\left\lbrace \begin{array}{c}
	\frac{\varepsilon}{2E_f}\hat{\ve u(k)}, \quad \text{if} 0<k<k_f\\
	0,\quad \text{otherwise}
	\end{array}\right. 
\end{equation}
程序中的默认加力形式,其中默认加力波数范围1$\sim$2,$\frac{\varepsilon}{2E_f}=0.5$
\noindent \textbf{线性加力}\\
在所有尺度上添加能量,此方法常用于有限差分方法,即\citep{lundgren_linearly_2003,rosales_linear_2005},
\begin{equation}
\hat{\ve f}=\frac{\varepsilon}{2E}\hat{\ve u}
\end{equation}
其中让$\frac{\varepsilon}{2E}$设置为经验常数,比如0.1333.
\subsubsection{常见统计量}
\begin{tabular}{cc}
		\hline 
		统计量&计算表达式\\
	\hline 
	湍动能&$E=\frac{1}{2}\langle\ve u\cdot \ve u\rangle=\frac{3}{2}u'^2=\int_0^{k_{max}}E(k)dk $  \\ 
	单位质量的平均能量耗散率 &$\varepsilon=2\nu \int_{0}^{k_{max}}k^2E(k)dk$ \\  
	积分长度 &$L=\frac{\pi}{2u'^2}\int_{0}^{k_{max}}k^{-1}E(k)dk$  \\ 
	泰勒微尺度 &$\lambda=(\frac{15\nu u'^2}{\varepsilon})^{1/2}$     \\ 
	Kolmogorov 长度尺度 &$\eta=(\frac{\nu^3}{\varepsilon})^{1/4}$     \\ 
	Eddy turnover time   &$T=L/u'$  \\ 
	Kolmogorov 时间尺度 &$\tau_\eta=(\frac{\nu}{\varepsilon})^{1/2}$\\
	基于泰勒微尺度的雷诺数&$Re_\lambda=\frac{u'\lambda}{\nu}$\\
	基于积分尺度与u'的能量耗散率&$a=\frac{\varepsilon L}{u'^3}$\\
	\hline 
	
\end{tabular} 

\subsection{均匀剪切湍流}
\subsubsection{基本方程}
$\ve u =\ve U+\ve u'$
平均速度为:
$U_1=0$, $U_2=Sx_1$,和 $U_3=0$.
\begin{equation}
\begin{split}
\frac{\partial  {u'}}{\partial t}+Sx_1\frac{\partial u_i'}{\partial x_2}+S\delta_{i2}{u'_1}+u'_j\frac{\partial u'_i}{\partial x_j} &=-\frac{1}{\rho} \frac{\partial p}{\partial x_i}+\nu \frac{\partial^2 u'_i}{\partial x_j\partial x_j}\\
\frac{\partial u'_i}{\partial x_i}&=0
\end{split}
\end{equation}
其中$\ve u'$是湍流脉动速度, $p$ 是压强脉动, $\nu$ 是运动粘度, $S$是平均剪切率.
\subsection{计算方法}
\begin{itemize}
	\item 标准计算域:$2\pi\times2\pi\times2\pi$ 或者 $2\pi\times4\pi\times2\pi$ (可修改)
	\item 空间格式: 伪谱法
	\item  时间格式:二阶显示Adams-Bashforth 格式
\end{itemize}
\subsection{剪切周期边界条件}
对于当前的设置,剪切平面在$x$-$y$平面,其中$x$是剪切周期方向。那么对于任意物理量,满足:
\begin{equation}
\phi(x,y,z)=\phi(x+L_x,y+L_xSt,z)
\end{equation}
在计算的过程中,如果需要时刻保证剪切周期边界条件,需要对网格进行变换。
\citet{rogallo_numerical_1981}通过对网格的变换,将网格转换到三轴都是周期边界条件的网格中,此时便可以对三个方向同时进行傅里叶变换。但这种变换并不能保证长时间有效的计算。在计算过程中,需要不断地对网格进行重新变换。如果为了避免网格变换,可以采用\citet{brucker_efficient_2007}的方法(本程序默认方法)。该方法通过在傅里叶变换中进行相位平移的方法避免了对网格的重新变换。具体变换过程如下:
\subsubsection*{FFT正变换}
\textbf{Step1}: 先对周期方向($y$与$z$方向)进行傅里叶变换。
\begin{equation}
	\check{\phi}(x,k_y,k_z)=\sum_{k_y} \sum_{k_z}\phi(x,y,z)\exp[-i(k_yy+k_zz)]
\end{equation}
\qquad\textbf{Step2}: 在x方向进行相位平移。
\begin{equation}
\tilde{\phi}(x,k_y,k_z)=\check{\phi}(x,k_y,k_z)\exp[iStk_yx]
\end{equation}
\qquad\textbf{Step3}: 做x方向的傅里叶变换。
\begin{equation}
\hat{\phi}(k_x,k_y,k_z)=\sum_{k_x}\tilde{\phi}(x,k_y,k_z)\exp[-ik_xx]
\end{equation}
同时,用$\mathcal{F}\left\lbrace \right\rbrace $表示这一过程。
\subsubsection*{FFT逆变换}
\textbf{Step1}: 先对$x$方向进行傅里叶逆变换。
\begin{equation}
\tilde{\phi}(x,k_y,k_z)=\frac{1}{N_x}\sum_{k_x}\hat\phi(k_x,k_y,k_z)\exp[i(k_xx)]
\end{equation}
\qquad\textbf{Step2}: 在x方向进行相位平移。
\begin{equation}
\check{\phi}(x,k_y,k_z)=\tilde{\phi}(x,k_y,k_z)\exp[-iStk_yx]
\end{equation}
\qquad\textbf{Step3}: 做y,z方向的傅里叶逆变换。
\begin{equation}
{\phi}(x,y,z)=\frac{1}{N_yN_z}\sum_{k_y} \sum_{k_z}\check{\phi}(x,k_y,k_z)\exp[i(k_yy+k_zz)]
\end{equation}
\subsubsection*{谱空间的波数}
在均匀各向同性湍流中, 在谱空间的波数由$k_x,k_y$与$k_z$表示,但对于均匀剪切湍流,波数$k_i$代表的是网格转到三个方向都是周期时的波数。并不是实际网格中的波数情况。此时,我们通过$k_i'=k_i-Stk_y\delta_{ix}$,此时可以认为在真实剪切湍流问题中的波数。其中,梯度算子和laplace算子都与各向同性湍流一样,只不过将$k_i$统一用$k_i'$替换即可。
\subsubsection*{谱空间中的NS方程}
时间导数:
\begin{equation}
\mathcal{F}\left\lbrace\frac{\partial \hat{u}_i'}{\partial t}\right\rbrace=\frac{\partial u_i'}{\partial t}-ISk_yx\hat{u}_i
\end{equation}
\qquad 空间梯度:
\begin{equation}
\mathcal{F}\left\lbrace\frac{\partial u_j'}{\partial x_i}\right\rbrace=Ik_i'\hat{u}_j.
\end{equation}
为了方便对比均匀各向同性湍流与均匀剪切湍流在谱空间中的不同,现将谱空间中的不可压缩NS方程表达如下:
\begin{equation}
\frac{\partial \hat{u}_i'}{\partial t}-Ik_j'\mathcal{F}\left\lbrace{u_i'u_j'}\right\rbrace=-Ik_i'\hat{p}^*-\nu k^2 \hat{u}_i-S\delta_{iy}\hat{u}_x\end{equation}
\subsection{时间分裂形式}
与各向同性湍流的类似,在此不再赘述。
\subsection{流场初始化方法}
与各向同性湍流的类似,在此不再赘述。

\subsection{常见统计量}
基本与各向同性湍流保持不变,但也引入一些与剪切率相关的统计量

\begin{tabular}{cc}
	\hline 
	统计量&计算表达式\\
	\hline 
	湍动能&$E=\frac{1}{2}\langle\ve u\cdot \ve u\rangle=\frac{3}{2}u'^2=\int_0^{k_{max}}E(k)dk $  \\ 
	单位质量的平均能量耗散率 &$\varepsilon=2\nu \int_{0}^{k_{max}}k^2E(k)dk$ \\  
	积分长度 &$L=\frac{\pi}{2u'^2}\int_{0}^{k_{max}}k^{-1}E(k)dk$  \\ 
	泰勒微尺度 &$\lambda=(\frac{15\nu u'^2}{\varepsilon})^{1/2}$     \\ 
	Kolmogorov 长度尺度 &$\eta=(\frac{\nu^3}{\varepsilon})^{1/4}$     \\ 
	Eddy turnover time   &$T=L/u'$  \\ 
	Kolmogorov 时间尺度 &$\tau_\eta=(\frac{\nu}{\varepsilon})^{1/2}$\\
	基于泰勒微尺度的雷诺数&$Re_\lambda=\frac{u'\lambda}{\nu}$\\
	基于积分尺度与u'的能量耗散率&$a=\frac{\varepsilon L}{u'^3}$\\
	\hline 
	
\end{tabular} 

\section{槽道湍流}
\begin{figure}[h]
		\centering
	\includegraphics[width=0.7\linewidth]{figure/part1_channel}
	\caption{槽道湍流示意图}
	\label{fig:part1channel}
\end{figure}
\subsection{基本方程}

\begin{equation}
\begin{split}
\frac{\partial  {u}}{\partial t}+u'_j\frac{\partial u'_i}{\partial x_j} &=-\frac{1}{\rho} \frac{\partial p}{\partial x_i}+\nu \frac{\partial^2 u_i}{\partial x_j\partial x_j}\\
\frac{\partial u_i}{\partial x_i}&=0
\end{split}
\end{equation}
其中$\ve u$是流体速度, $p$ 是压强, $\nu$ 是运动粘度.
\subsection{计算方法}
\begin{itemize}
	\item 标准计算域:$2\pi\times2h\times\pi$ (可修改)
	\item 空间格式: $x,z$方向伪谱法,$y$方向二阶中心差分。
	\item  时间格式:二阶显示Adams-Bashforth 格式
	\item 支持定流量(程序默认)或定压力梯度的Poiseuille 流动
\end{itemize}

\subsection{时间分裂形式}
与各向同性湍流的类似,在此不再赘述。
\subsection{流场初始化方法}
初始时刻,采用平均速度$+$上下壁面条带$+$随机扰动的方式。

平均速度剖面:
\begin{equation}
U(y^+)=2.5\log_{10}{y^+}+5.5
\end{equation}

壁面条带扰动:

\begin{equation}
\begin{split}
u_s(i,j,k)&=\frac{\alpha}{\delta}\sqrt{e}y^+e^{-\frac{1}{\delta^2}{y^+}^2}\cos{\frac{L_z^+}{l_z^+}\frac{2\pi k}{N_z}}\\
v_s(i,j,k)&=0;\\
w_s(i,j,k)&=\frac{\alpha}{\delta}\sqrt{e}y^+e^{-\frac{1}{\delta^2}{y^+}^2}\cos{\frac{L_x^+}{l_x^+}\frac{2\pi i}{N_x}}.\\
\end{split}
\end{equation}
这里的参数$\alpha=15$,$\delta=50$,$l_x=300$,$l_z=100$.这些值均为经验参数,且可能与一些文献有出入,但不影响湍流最终的发展。

\section{基于贴体网格的槽道求解器}
\begin{figure}[h]
	\centering
	\includegraphics[width=0.7\linewidth]{figure/part1_moving_channel}
	\caption{波形壁槽道湍流示意图}
	\label{fig:part1movingchannel}
\end{figure}
\begin{equation}
t=\tau, x_1=\xi_1, x_2=(\xi_2-1)(1+\eta_-)+(\eta_+ +1),x_3=\xi_3
\end{equation}
其中

\begin{equation}
\eta_-=\frac{1}{2}(\eta_u-\eta_b), \eta_+ =\frac{1}{2}(\eta_u+\eta_b)
\end{equation}
经过坐标变换后, 上下边界在新坐标下依然位于$\xi_2=0,1$处。
物理空间与计算空间之间的微分关系:
\begin{equation}
\begin{split}
\Partial{}{t}&=\Partial{}{\tau}\Partial{\tau}{t}+\Partial{}{\xi_2}\Partial{\xi_2}{t}=\Partial{}{\tau}+\phi_t\Partial{}{\xi_2}\\
\Partial{}{x_1}&=\Partial{}{\xi_1}\Partial{\xi_1}{x_1}+\Partial{}{\xi_2}\Partial{\xi_2}{x_1}=\Partial{}{\xi_1}+\phi_1\Partial{}{\xi_2}\\
\Partial{}{x_2}&=\Partial{}{\xi_2}\Partial{\xi_2}{x_2}=\Partial{}{\xi_2}+\phi_2\Partial{}{\xi_2}\\
\Partial{}{x_3}&=\Partial{}{\xi_3}\Partial{\xi_3}{x_3}+\Partial{}{\xi_2}\Partial{\xi_2}{x_3}=\Partial{}{\xi_3}+\phi_3\Partial{}{\xi_2}\\
\end{split}
\end{equation}
其中
\begin{equation}\label{eq:phi_i}
\begin{split}
\phi_t&=\Partial{\xi_2}{t}=-\frac{1}{\eta_-+1}[(\xi_2-1)\Partial{\eta_-}{t}+\Partial{\eta_+}{t}]\\
\phi_1&=\Partial{\xi_2}{x_1}=-\frac{1}{\eta_-+1}[(\xi_2-1) \Partial{\eta_-}{x_1}+\Partial{\eta_+}{x_1}]\\
\phi_2&=\Partial{\xi_2}{x_2}=\frac{1}{\eta_-+1}-1\\
\phi_3&=\Partial{\xi_2}{x_3}=-\frac{1}{\eta_-+1}[(\xi_2-1) \Partial{\eta_-}{x_3}+\Partial{\eta_+}{x_3}]\\
\end{split}
\end{equation}
\iffalse
Laplace 算子为:
\begin{equation}
\begin{split}
\nabla^2 =\frac{\partial ^2}{\partial x_j \partial x_j}=&\frac{\partial ^2}{\partial \xi_1^2}+\frac{\partial ^2}{\partial \xi_3^2}\\
&+(\Partial{\phi_1}{\xi_1}+\Partial{\phi_3}{\xi_3})\Partial{}{\xi_2}\\
&+2(\phi_1\Partial{}{\xi_1}+\phi_3\Partial{}{\xi_3})\Partial{}{\xi_2}\\
&+\frac{1}{2}\Partial{\Phi}{\xi_2}\Partial{}{\xi_2}+\Phi\frac{\partial ^2}{\partial \xi_2^2}\\
\end{split}
\end{equation}
其中 $\Phi=\phi_1^2+\phi_2^2+\phi_3^2$

非线性项的处理,在计算空间进行
\fi
具体的原理详见《运动边界槽道湍流数值模拟方法》\footnote{原说明文档中$\eta_+, \eta_-$ 分别用$\eta_0, \eta$表示,公式(\ref{eq:phi_i})中的$\phi_2$在原说明文档中有笔误。}





\part{颗粒部分}
\chapter{颗粒求解器基本理论}


\section{球形颗粒}
\begin{equation}
\begin{split}
m_p \frac{d v_i}{dt}&=F_i^d+F_i^{Saff}\pm (m_p-m_f)g_i\\
I_p \frac{d \omega_i}{dt}&=M_i^d
\end{split}
\end{equation}

\subsection{阻力模型}
\begin{itemize}
\item Stokes阻力: $\ve F^d=6\pi \mu R (\ve u-\ve v)$.
\item 经验阻力公式:$\ve F^d=6\pi \mu R (\ve u-\ve v)(1+0.15Re_p^{0.687})$.
\end{itemize}
\subsection{Saffman 升力}
\begin{equation}
\ve F^{Saff}=1.615J\mu2R|\Delta \ve u|\sqrt{\frac{4R^2|\ve \Omega|}{\nu}}\frac{\ve \Omega\times\Delta \ve u}{|\ve \Omega||\Delta \ve u|}
\end{equation}
 其中$\Delta \ve u=\ve v-\ve u$.
\begin{itemize}
\item 	对于Saffman model 1: $J=1$
\item	对于Saffman model 2: $J=0.3(1+\tanh[2.5(\log_{10}\epsilon+0.191)])(\frac{2}{3}+\tanh(6\epsilon-1.92))$, 其中$\epsilon=\sqrt{|\ve \Omega|\nu}{\Delta \ve u}$.
\end{itemize}
\section{椭球颗粒}
\subsection{非球形颗粒绕流Stokes解的一般形式}
对于一般非球形的颗粒的绕流问题,其颗粒的受力可以描述为阻力,力矩与应力子。 它们的表达式分别为:
\begin{equation}{\label{eq:part01-stokes}}
\begin{split}
F_i &= K_{ij}^t(U_j-v_j)+K_{ij}^c(\Omega_j-\omega_j)+\Phi_{ijk}S_{jk}\\ 
M_i &= K_{ij}^c(U_j-v_j)+K_{ij}^r(\Omega_j-\omega_j)+\Theta_{ijk}S_{jk}\\
\Sigma_{ij}&=\Phi_{kij}(U_k-v_k)+\Theta_{kij}(\Omega_k-\omega_k)+Q_{ijkl}S_{kl}\\
\end{split}
\end{equation}
其中$K_{ij}^t, K_{ij}^C,K_{ij}^r,\Theta_{ijk},\Phi_{ijk},Q_{ijkl}$均与颗粒形状有关。同时,由于应力子的对称性,这些系数满足一些对称特点,即
\begin{equation}
\begin{split}
K_{ij}^t&=K_{ji}^t,\quad K_{ij}^c=K_{ji}^c,\quad K_{ij}^r=K_{ji}^r,\\ 
\Phi_{ijk}&=\Phi_{ikj},\quad \Theta_{ijk}=\Theta_{ikj},\\
Q_{ijkl}&=Q_{ijlk},\quad Q_{ijkl}=Q_{ijlk},\quad Q_{ijkl}=Q_{klij},\quad Q_{ijkl}=Q_{jilk}.\\
\end{split}
\end{equation}
通过公式(\ref{eq:part01-stokes})可知,对于一般形状的颗粒,颗粒在流场中受到的合力,合力矩是相互耦合,且应力子也与颗粒的平动相关。但对于中心对称颗粒,$K_{ij}^c=0$, $\phi_{ijk}=0$. 即:
   	\begin{equation}
\begin{split}
F_i &= K_{ij}^t(U_j-v_j)\\ 
M_i &= K_{ij}^r(\Omega_j-\omega_j)+\Theta_{ijk}S_{jk}\\
\Sigma_{ij}&=\Theta_{kij}(\Omega_k-\omega_k)+Q_{ijkl}S_{kl}.\\
\end{split}
\end{equation}
理论上,如果我们能获得任意形状颗粒的这些参数的具体表达式或者数值,我们便可以将其用于点颗粒计算。
\subsection{姿态描述}
对于非球形颗粒,颗粒的取向需要考虑。在此,我们引入颗粒的姿态描述。在介绍姿态描述之前,我们需要定义惯性坐标系,随体坐标系和颗粒坐标系,并阐述两种坐标系之间的联系。在本程序中,默认使用椭球颗粒模型。
\begin{figure}[h]
		\centering
	\includegraphics[width=0.8\linewidth]{Part1_frame.png}
	\caption{坐标系示意图}
\end{figure}
基矢量转换关系:
$\ve g_{i'}=R_{i'i} \ve g_{i}$, $\ve g_i=R_{ii'} \ve g_{i'}$ 。 其中 $\ve g_i$ 惯性坐标系的基矢量,
$\ve g_{i'}$  颗粒坐标系的基矢量。\\
 将惯性空间的矢量在颗粒坐标系中描述: $\ve n= n_i \ve g_i =n_i R_{ii'} \ve g_{i'}$ \\ 
 将颗粒空间的矢量在惯性坐标系中描述:$\ve n'= n_{i'} \ve g_{i'} =n_{i'} R_{i'i} \ve g_{i}$\\
 将惯性空间的张量在颗粒坐标系中描述: $\ma K= K_{ij} \ve g_i \ve g_j =K_{ij} R_{ii'}R_{jj'} \ve g_{i'}\ve g_{j'}$ \\ 
 将颗粒空间的张量在惯性坐标系中描述:$\ma K'= K_{i'j'} \ve g_{i'} \ve g_{j'} =K_{i'j'} R_{i'i}R_{j'j} \ve g_{i}\ve g_{j}$\\
\subsubsection{欧拉角与卡丹角}
\subsubsection{欧拉四元数}
在数值计算中,欧拉角和卡丹角容易产生奇异性。 但我们可以通过欧拉四元数来表征颗粒的取向。对于欧拉四元数的数学原理详见\obs{REF}
\begin{equation}
e_{0}^{2}+e_{1}^{2}+e_{2}^{2}+e_{3}^{2}=1
\end{equation}
\begin{equation}
\ma R_{i'i}=\left[ \begin{array}{ccc}{e_{0}^{2}+e_{1}^{2}-e_{2}^{2}-e_{3}^{2}} & {2\left(e_{1} e_{2}+e_{0} e_{3}\right)} & {2\left(e_{1} e_{3}-e_{0} e_{2}\right)} \\ {2\left(e_{1} e_{2}-e_{0} e_{3}\right)} & {e_{0}^{2}-e_{1}^{2}+e_{2}^{2}-e_{3}^{2}} & {2\left(e_{2} e_{3}+e_{0} e_{1}\right)} \\ {2\left(e_{1} e_{3}+e_{0} e_{2}\right)} & {2\left(e_{2} e_{3}-e_{0} e_{1}\right)} & {e_{0}^{2}-e_{1}^{2}-e_{2}^{2}+e_{3}^{2}}\end{array}\right]
\end{equation}

$\ve e =[e_0~e_1~e_2~e_3]^T$
\begin{equation}
\mathbf{G}=\left[ \begin{array}{cccc}{-e_{1}} & {e_{0}} & {e_{3}} & {-e_{2}} \\ {-e_{2}} & {-e_{3}} & {e_{0}} & {e_{1}} \\ {-e_{3}} & {e_{2}} & {-e_{1}} & {e_{0}}\end{array}\right]
\end{equation}
\begin{equation}
\dot{\mathbf{e}}=\frac{1}{2} \mathbf{G}^{T} \boldsymbol{\omega}'
\end{equation}

\subsection{平动模型}
对于一般性的椭球颗粒,其平动方程为
\begin{equation}\label{eq:ch2-etranslation}
m_{p} \frac{d v_{i}}{d t}= \mu  K_{i j} \Delta u_{j} \pm (m_p-m_f)\ve g
\end{equation}
其中$K_{i j}=R_{i i'}K_{i' j'} R_{j' j}$,取决于颗粒的取向. 其任意椭球体的Stokes
阻力模型\citep{brenner_stokes_1963}:
\begin{equation}
\mathbf{K'}=16 \pi\left[\frac{\ve {g}_{1} \ve {g}_{1}}{\chi+a_{1}^{2} \alpha_{1}}+\frac{\ve{g}_{2} \ve{g}_{2}}{\chi+a_{2}^{2} \alpha_{2}}+\frac{\ve{g}_{3} \ve{g}_{3}}{\chi+a_{3}^{2} \alpha_{3}}\right]
\end{equation}
为了方便表述,$\ve g_i$代表固结在颗粒坐标系上的基矢量。其中$a_i$($i=1,2,3$) 分别代表椭球三个主轴的半轴长,而$\alpha_{i}$和$\chi$均代表椭球体的形状因子。其各自的表达式为
\begin{equation}
\chi=\int_{0}^{\infty} \frac{d \lambda}{\Delta(\lambda)},
\end{equation}
\begin{equation}
\alpha_{i}=\int_{0}^{\infty} \frac{d \lambda}{\left(a_{j}^{2}+\lambda\right) \Delta(\lambda)} \quad(j=1,2,3).
\end{equation}
其中
\begin{equation}
\Delta(\lambda)=\sqrt{\left(a_{1}^{2}+\lambda\right)\left(a_{2}^{2}+\lambda\right)\left(a_{3}^{2}+\lambda\right)},
\end{equation}
将平动模型化简后,
\begin{equation}
 \frac{d v_{i}}{d t}= \frac{12 \nu}{D r^2 k_1k_2k_3}  \tilde{K}_{i' j'}R_{i' i} R_{j'j} \Delta u_{j} \pm (1-\frac{1}{D}) g_i
\end{equation}
其中$D$为颗粒与流体的密度比$\rho_p/\rho_f$,$k_i$为颗粒的长细比,即 $k_i=a_i/r$。 $r$为参考半径。
而无量纲阻力矩阵$\mathbf{\tilde{K}'}$为
\begin{equation}
\mathbf{\tilde{K}'}=\left[\frac{\ve{g}_{1} \ve{g}_{1}}{\tilde{\chi}+k_{1}^{2} \tilde{\alpha}_{1}}+\frac{\ve{g}_{2} \ve{g}_{2}}{\tilde{\chi}+k_{2}^{2} \tilde{\alpha}_{2}}+\frac{\ve{g}_{3}\ve{g}_{3}}{\tilde{\chi}+k_{3}^{2} \tilde{\alpha}_{3}}\right]
\end{equation}

而进行归一化的形状因子分别表示为:
\begin{equation}
\begin{split}
\tilde {\alpha}&=\int_{0}^{\infty} \frac{d \lambda}{\Delta(\lambda)(k_1^2+\lambda)},\quad \tilde {\beta}=\int_{0}^{\infty} \frac{ d \lambda}{\Delta(\lambda)(k_2^2+\lambda)};\\
\tilde {\gamma}&=\int_{0}^{\infty} \frac{d \lambda}{\Delta(\lambda)(k_3^2+\lambda)},\quad \tilde {\chi}=\int_{0}^{\infty} \frac{ d \lambda}{\Delta(\lambda)};\\
\end{split}
\end{equation}

对于回转体的椭球颗粒,令$r=a_1$, $a_1=a_2$.则有$k_1=k_2=1, k_3=\lambda$ (这里的$\lambda$ 指的长轴与短轴之比).

此时椭球体的形状参数有解析解:

\begin{tabular}{|c|c|c|c|}
	\hline 
	& 碟状颗粒 & 球形颗粒 & 杆状颗粒\\
	\hline
	& $\lambda<1$ & $\lambda=1$ & $\lambda>1$ \\ 
	\hline 
	$\tilde{\alpha}_1=\tilde{\alpha}_2$&$-\frac{B-\pi}{2(1-\lambda^2)^{3/2}}-\frac{\lambda}{(1-\lambda^2)}$& 2/3 &  $\frac{-A}{2(\lambda^2-1)^{3/2}}+\frac{\lambda}{(\lambda^2-1)}$\\ 
	\hline 
	$\tilde{\alpha}_3$& $\frac{B-\pi}{(1-\lambda^2)^{3/2}}+\frac{2}{(1-\lambda^2)\lambda}$ &2/3&  $\frac{A}{(\lambda^2-1)^{3/2}}-\frac{2}{\lambda(\lambda^2-1)}$ \\ 
	\hline 
	$\tilde{\chi}$&$-\frac{B-\pi}{(1-\lambda^2)}$&2& $\frac{A}{\sqrt{\lambda^2-1}}$  \\ 		
	\hline 
\end{tabular} 

其中 $A=2\ln {(\lambda+\sqrt{\lambda^2-1})}$ 和 $B=2\arctan{\frac{\lambda}{\sqrt{1-\lambda^2}}}$

\subsection{转动模型}

椭球颗粒转动动力学模型\citep{jeffery_motion_1922,lundell_heavy_2010,lundell_effect_2011}:

\begin{equation}
\ma{I'} \cdot \dot{\ve\omega'}+ \ve \omega' \times (\ma I' \cdot \ve \omega')=\ve M'
\end{equation}
其中$\prime$代表颗粒坐标系。颗粒受到的合外力矩为$\ve M'$
\begin{equation}
	\begin{split}
M_{1'}=\frac{16\pi\mu}{3(a_2^2 \alpha_2+a_3^2 \alpha_3)}\left[ (a_2^2-a_3^2)S_{2'3'}+(a_2^2+a_3^2)(\Omega_{3'2'}-\omega_{1'})\right] \\
M_{2'}=\frac{16\pi\mu}{3(a_3^2 \alpha_3+a_1^2 \alpha_1)}\left[ (a_3^2-a_1^2)S_{1'3'}+(a_3^2+a_1^2)(\Omega_{1'3'}-\omega_{2'})\right] \\
M_{3'}=\frac{16\pi\mu}{3(a_1^2 \alpha_1+a_2^2 \alpha_2)}\left[ (a_1^2-a_2^2)S_{1'2'}+(a_1^2+a_2^2)(\Omega_{2'1'}-\omega_{3'})\right] \\
	\end{split}
\end{equation}
写成基于角速度的微分方程形式
\begin{equation}
	\begin{split}
\frac{d \omega_{1'}}{d t}=\frac{20 \nu}{D r^2 k_1k_2k_3}\tilde{M}_{1'}-\Theta_1 \omega_{2'}\omega_{3'}\\
\frac{d \omega_{2'}}{d t}=\frac{20 \nu}{D r^2 k_1k_2k_3}\tilde{M}_{2'}-\Theta_2 \omega_{1'}\omega_{3'}\\
\frac{d \omega_{3'}}{d t}=\frac{20 \nu}{D r^2 k_1k_2k_3}\tilde{M}_{3'}-\Theta_3 \omega_{1'} \omega_{2'}\\
	\end{split}
\end{equation}
为此,定义新的形状参数为$\Theta_1=\frac{k_2^2-k_3^2}{k_2^2+k_3^2}$,$\Theta_2=\frac{k_3^2-k_1^2}{k_3^2+k_1^2}$,$\Theta_3=\frac{k_1^2-k_2^2}{k_1^2+k_2^2}$,则对应的无量纲力矩为
\begin{equation}
\begin{split}
\tilde{M}_{1'}&=\frac{1}{(k_2^2 \tilde{\alpha}_2+k_3^2 \tilde{\alpha}_3)}\left[ \Theta_1 S_{2'3'}+(\Omega_{3'2'}-\omega_{1'})\right] \\
\tilde{M}_{2'}&=\frac{1}{(k_3^2 \tilde{\alpha}_3+k_1^2 \tilde{\alpha}_1)}\left[ \Theta_2 S_{1'3'}+(\Omega_{1'3'}-\omega_{2'})\right] \\
\tilde{M}_{3'}&=\frac{1}{(k_1^2 \tilde{\alpha}_1+k_2^2 \tilde{\alpha}_2)}\left[ \Theta_3 S_{1'2'}+(\Omega_{2'1'}-\omega_{3'})\right] \\
\end{split}
\end{equation}
\subsection{颗粒响应时间}
考虑颗粒取向随机,定义颗粒的平动响应时间为\citep{brenner_stokes_1963,shin_rotational_2005}
\begin{equation}
\tau_p=\frac{Dr^2k_1k_2k_3}{12\nu}\frac{1}{\bar{\tilde{K}}}
\end{equation}
其中$\bar{\tilde{K}}$
\begin{equation}
\bar{\tilde{K}}=3(\tilde{K}_{1'1'}^{-1}+\tilde{K}_{2'2'}^{-1}+\tilde{K}_{3'3'}^{-1})^{-1}
\end{equation}
对于回转椭球体,其颗粒响应时间有解析解\citep{shapiro_deposition_1993}(注:$\lambda=a/b$, 其中$a$为长半轴,$b$为短半轴。表达式取决于颗粒的长细比定义方式。)
\begin{equation}
\tau_{p,prolate}=\frac{Da^2\lambda}{9\nu}\frac{2\ln{(\lambda+\sqrt{\lambda^2-1})}}{\sqrt{\lambda^2-1}}
\end{equation}
\begin{equation}
\tau_{p,oblate}=\frac{Da^2\lambda}{9\nu}\frac{\pi-2\arctan{\frac{\lambda}{\sqrt{1-\lambda^2}}}}{\sqrt{1-\lambda^2}}
\end{equation}
目前,暂时没有颗粒转动的响应时间的估计公式。
\section{弱惯性椭球颗粒模型}
当颗粒尺寸和密度足够小(大于或等于流体密度), 一般的颗粒运动模型的加速项计算十分困难。 在此,我们认为颗粒的合外力为零。因此,可以得出弱惯性椭球颗粒的运动控制方程。
\subsection{平动模型}
当$\frac{d v}{dt}=0$, 方程~(\ref{eq:ch2-etranslation})为
\begin{equation}
 \mu  K_{i j} \Delta u_{j} \pm (m_p-m_f)\ve g = 0
\end{equation}
因此,颗粒的速度将由两部分组成
\begin{equation}
\begin{split}
\ve v &=\ve u+\ve v_g\\
\ve v_g&=\pm\frac{D r^2 k_1k_2k_3}{12 \nu}(1-\frac{1}{D}) \ma {K}^{-1} \ve g
\end{split}
\end{equation}
其中$\ve v_g$为 Stokes 沉降速度, $\ve u$ 为颗粒处的流体速度。当不考虑沉降时,颗粒完全跟随流体。
\subsection{转动模型}
同理, 让$\frac{d \ve \omega'}{dt}=0$, 颗粒的转动速度为
\begin{equation}
\begin{split}
\omega_{1'}&=\Theta_1 S_{2'3'}+\Omega_{3'2'} \\
\omega_{2'}&=\Theta_2 S_{1'3'}+\Omega_{1'3'} \\
\omega_{3'}&=\Theta_3 S_{1'2'}+\Omega_{2'1'} \\
\end{split}
\end{equation}
\section{游动颗粒模型}

\section{pFTLE颗粒模型}
此类颗粒的类型主要用来计算随流体轨迹线的李雅普诺夫指数和拉格朗日拉伸与压缩方向。本颗粒不直接求解变形梯度张量,而是直接求解左柯西格林张量的特征向量与特征值进行近似计算。具体的概念与实现过程,可见Cui et al.(Draft).在此,仅简要给出公式。
	\begin{equation}\label{eq:angular velocity}
\begin{split}
\tilde{\omega}_1&= \tilde{\Omega}_{1} +\epsilon_1\tilde{S}_{23},\\
\tilde{\omega}_2&= \tilde{\Omega}_{2}+ \epsilon_2\tilde{S}_{31},\\
\tilde{\omega}_3&= \tilde{\Omega}_{3} +\epsilon_3\tilde{S}_{12}.
\end{split}
\end{equation}
其中
	\begin{equation}
\epsilon_1=\frac{e^{2\rho_2}+e^{2\rho_3}}{e^{2\rho_2}-e^{2\rho_3}},\epsilon_2=\frac{e^{2\rho_3}+e^{2\rho_1}}{e^{2\rho_3}-e^{2\rho_1}},\epsilon_3=\frac{e^{2\rho_1}+e^{2\rho_2}}{e^{2\rho_1}-e^{2\rho_2}}.
\end{equation}
而各个向量方向上的特征值的时间推进为:
\begin{equation}{\label{eq:rho_t}}
\frac{d \rho_i}{d t} = \tilde{S}_{ii}
\end{equation}
\section{颗粒处流场信息插值方法}
\subsection{三维二阶拉格朗日插值}
本程序的颗粒处的插值函数使用的是二阶拉格朗日插值。(注:在算有壁面问题时,壁面附近的网格不能太密,比如用$\cos\frac{2\pi j}{N}$,此网格在算高雷诺数问题时,插值容易出现龙格现象。)
\subsubsection{均匀网格方向}
\subsubsection{非均匀网格方向}
\section{时间推进方法}
\subsection{二阶显式Adams-Bashforth方法}
对于常微分方程组$\dot{f}=g$。 
\begin{equation}
\frac{f^{n+1}-f^{n}}{\Delta t}=\frac{3}{2}g^{n}-\frac{1}{2}g^{n-1}
\end{equation}

\section{颗粒碰撞}


\part{流固耦合}
%\chapter{颗粒与流场双向耦合基本理论}

\section{力耦合}
\section{应力与力矩耦合}
根据\citet{batchelor_stress_1970}的推导\footnote{原文中的Jeffery力矩在左手系,本文推导一律右手系,因此力矩与原文相差一个负号},单个颗粒产生的颗粒应力的表达式为:
\begin{equation}\label{eq:ch3-stress}
\begin{split}
\sigma_{ij}^{(p)}&=\frac{4\pi\mu}{V_{box}} \frac{1}{2}(H_{ij}+H_{ji})+\frac{1}{2V_{box}} \epsilon_{ijk} M_k\\
&=\frac{1}{V_{box}}{ [ C_{ijkl}S_{kl}+C_{ijk} M_k -\frac{1}{2}\epsilon_{ijk} M_k]}
\end{split}
\end{equation}
其中$V_{box}$是欧拉网格体积,$M_k$是作用在颗粒上的合力矩(Jeffery 力矩)\footnote{与\citet{batchelor_stress_1970}文章中差负号}。将公式(\ref{eq:ch3-stress})转入到颗粒坐标系, 即得
\begin{equation}
\sigma_{i'j'}^{(p)}
=\frac{1}{V_{box}}{ [C_{i'j'k'l'}S_{k'l'}+C_{i'j'k'} M_{k'}-\frac{1}{2}\epsilon_{i'j'k'}M_{k'}]}
\end{equation}
各项的表达式为:
\begin{equation}
\begin{split}
 C_{i'j'k'l'}S_{k'l'}&=\frac{16\pi \mu abc}{3(J_1J_2+J_2J_3+J_3J_1)}\left[\begin{matrix}
 J_1 S_{1'1'}-\Delta & &\\
 &J_2 S_{2'2'}-\Delta &\\
 & &J_3 S_{3'3'}-\Delta\\
 \end{matrix} \right] \\
 &+\frac{16\pi \mu abc}{3}\left[\begin{matrix}
  &S_{1'2'}/I_3&S_{1'3'}/I_2\\
 S_{2'1'}/I_3& &S_{2'3'}/I_1\\
 S_{3'1'}/I_2&S_{3'2'}/I_1 &\\
 \end{matrix} \right] \\
 \end{split}
\end{equation}

\begin{equation}
\begin{split}
C_{i'j'k'}  M_{k'}&=\frac{1}{2}\left[\begin{matrix}
&\Theta_3 M_{3'}&\Theta_2M_{2'}\\
\Theta_3M_{3'}& &\Theta_1 M_{1'}\\
\Theta_2M_{2'}&\Theta_1M_{1'} &\\
\end{matrix} \right] \\
\end{split}
\end{equation}
\begin{equation}
\begin{split}
\tau_{ij} =-\frac{1}{2}\epsilon_{i'j'k'} M_{k'}&=-\frac{1}{2}\left[\begin{matrix}
&M_{3'}&-M_{2'}\\
-M_{3'}& &M_{1'}\\
M_{2'}&-M_{1'} &\\
\end{matrix} \right] \\
\end{split}
\end{equation}
其中形状参数为$\Theta_1=\frac{k_2^2-k_3^2}{k_2^2+k_3^2}$,$\Theta_2=\frac{k_3^2-k_1^2}{k_3^2+k_1^2}$,$\Theta_3=\frac{k_1^2-k_2^2}{k_1^2+k_2^2}$。$\Delta=\frac{1}{3}(J_1S_{1'1'}+J_2S_{2'2'}+J_3S_{3'3'})$同时 $I_i$和$J_i$分别为:
\begin{equation}
\begin{split}
I_{1}=\int_{0}^{\infty} \frac{k_1k_2k_3(k_2^2+k_3^2)d \lambda}{\Delta(\lambda)(k_2^2+\lambda)(k_3^2+\lambda)},\quad J_{1}=\int_{0}^{\infty} \frac{k_1k_2k_3\lambda d \lambda}{\Delta(\lambda)(k_2^2+\lambda)(k_3^2+\lambda)};\\
I_{2}=\int_{0}^{\infty} \frac{k_1k_2k_3(k_1^2+k_3^2)d \lambda}{\Delta(\lambda)(k_1^2+\lambda)(k_3^2+\lambda)},\quad J_{2}=\int_{0}^{\infty} \frac{k_1k_2k_3\lambda d \lambda}{\Delta(\lambda)(k_1^2+\lambda)(k_3^2+\lambda)};\\
I_{3}=\int_{0}^{\infty} \frac{k_1k_2k_3(k_1^2+k_2^2)d \lambda}{\Delta(\lambda)(k_1^2+\lambda)(k_2^2+\lambda)},\quad J_{3}=\int_{0}^{\infty} \frac{k_1k_2k_3\lambda d \lambda}{\Delta(\lambda)(k_1^2+\lambda)(k_2^2+\lambda)};\\
\end{split}
\end{equation}
其中$M_{i'}$ 与$\tilde{M}_{i’}$的关系
\begin{equation}\label{key}
\begin{split}
M_{1'}=\frac{16\pi \mu r^3}{3}(k_2^2+k_3^2)\tilde{M}_{1'}\\
M_{2'}=\frac{16\pi \mu r^3}{3}(k_3^2+k_1^2)\tilde{M}_{2'}\\
M_{3'}=\frac{16\pi \mu r^3}{3}(k_1^2+k_2^2)\tilde{M}_{3'}\\
\end{split}
\end{equation}
其中$D$为密度比,$r$为参考半径,$k_i$为长细比。具体标记见颗粒求解器基本理论。



\bibliographystyle{jfm}
\bibliography{reference}% Produces the bibliography via BibTeX.
\end{document}
